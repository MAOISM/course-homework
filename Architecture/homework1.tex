%%%%%%%%%%%%%%%%%%%%%%%%%%%%%%%%%%%%%%%%%
% Short Sectioned Assignment
% LaTeX Template
% Version 1.0 (5/5/12)
%
% This template has been downloaded from:
% http://www.LaTeXTemplates.com
%
% Original author:
% Frits Wenneker (http://www.howtotex.com)
%
% License:
% CC BY-NC-SA 3.0 (http://creativecommons.org/licenses/by-nc-sa/3.0/)
%
%%%%%%%%%%%%%%%%%%%%%%%%%%%%%%%%%%%%%%%%%

%----------------------------------------------------------------------------------------
%	PACKAGES AND OTHER DOCUMENT CONFIGURATIONS
%----------------------------------------------------------------------------------------

\documentclass[paper=a4, fontsize=11pt]{scrartcl} % A4 paper and 11pt font size

\usepackage[T1]{fontenc} % Use 8-bit encoding that has 256 glyphs
\usepackage{fourier} % Use the Adobe Utopia font for the document - comment this line to return to the LaTeX default
\usepackage[english]{babel} % English language/hyphenation
\usepackage{amsmath,amsfonts,amsthm} % Math packages
\usepackage[CJKbookmarks=true,
            colorlinks,linkcolor=black,anchorcolor=blue,citecolor=green]{hyperref} 

\usepackage{lipsum} % Used for inserting dummy 'Lorem ipsum' text into the template

\usepackage{sectsty} % Allows customizing section commands
\allsectionsfont{\centering \normalfont\scshape} % Make all sections centered, the default font and small caps

\usepackage{fancyhdr} % Custom headers and footers
\pagestyle{fancyplain} % Makes all pages in the document conform to the custom headers and footers
\fancyhead{} % No page header - if you want one, create it in the same way as the footers below
\fancyfoot[L]{} % Empty left footer
\fancyfoot[C]{} % Empty center footer
\fancyfoot[R]{\thepage} % Page numbering for right footer
\renewcommand{\headrulewidth}{0pt} % Remove header underlines
\renewcommand{\footrulewidth}{0pt} % Remove footer underlines
\renewcommand\thesection{\roman{section}}

\setlength{\headheight}{13.6pt} % Customize the height of the header

\numberwithin{equation}{section} % Number equations within sections (i.e. 1.1, 1.2, 2.1, 2.2 instead of 1, 2, 3, 4)
\numberwithin{figure}{section} % Number figures within sections (i.e. 1.1, 1.2, 2.1, 2.2 instead of 1, 2, 3, 4)
\numberwithin{table}{section} % Number tables within sections (i.e. 1.1, 1.2, 2.1, 2.2 instead of 1, 2, 3, 4)

\setlength\parindent{0pt} % Removes all indentation from paragraphs - comment this line for an assignment with lots of text

%----------------------------------------------------------------------------------------
%	TITLE SECTION
%----------------------------------------------------------------------------------------

\newcommand{\horrule}[1]{\rule{\linewidth}{#1}} % Create horizontal rule command with 1 argument of height

\title{	
\normalfont \normalsize
\textsc{School of Software, Tsinghua University} \\ [25pt] % Your university, school and/or department name(s)
\horrule{0.5pt} \\[0.4cm] % Thin top horizontal rule
\huge Architecture of Computer and Network (2)\\ % The assignment title
\LARGE\textit{homework 1}
\horrule{2pt} \\[0.5cm] % Thick bottom horizontal rule
}

\author{Qingfu Wen \\ \normalsize 2011013239} % Your Info
\date{\normalsize\today} % Today's date or a custom date

\begin{document}

\maketitle % Print the title
\tableofcontents
\newpage
%----------------------------------------------------------------------------------------
%	PROBLEM 1
%----------------------------------------------------------------------------------------

\section{Problem 1}
Transform the following signed binary integer into decimal integer. \\
    $1) 10000000; \ 2) 11001100;  \ 3) 10110111 $ \\
\\
\begin{enumerate}
\item
for the MSB is $1$, it is a negative integer. The true form of it is $10000000$.
Thus, the decimal integer is $-128$ \\
\item
The true form of $11001100$ is $110100$. So, the decimal integer is $-52$.  \\
\item
The true form of $10110111$ is $1001001$. So, the decimal integer is $-73$.
\end{enumerate}

\section{Problem 2}
Transform the following signed decimal integer into hexadecimal integer(16bits). \\
$1) -32 \  2) -62$ \\
\\
\begin{enumerate}
\item
FFE0\\
\item
FFC3\\
\end{enumerate}

\section{Problem 3}
what are the main steps of instruction execution cycle? \\
\begin{enumerate}
\item Fetch
\item Decode
\item Fetch operands
\item Execute
\item Store output
\end{enumerate}

\section{Problem 4}
what are the range of memory address in real-address mode? \\
1M, 0x00000 $\sim$ 0xfffff\\

\section*{Problem 5}
Convert 0950:0100 to a linear in real-address mode\\
0950+0100 = 0A50 \\

\section{Problem 6}
List the 4 parts of the Assembly Language instructions\\
\begin{itemize}
 \item Label (optional)
 \item Mnemonic (required)
 \item Operand (depends on the instruction)
 \item Comment (optional)
\end{itemize}

\section{Problem 7}
Declare a 16 bit unsigned integer variable named 'wAarry' with 3 initial value.\\
wAarry WORD 1, 2, 3

\section{Problem 8}
Use equality sign directive to declare a constant sign, which equals 08h.\\
BACK = 08h

\section{Problem 9}
What are the three types of operand?\\
immediate, register, memory

\section{Problem 10}
Use assembly language to implement AX=(-val2+BX)-val4 \\
mov ax, -val2 \\
add ax, bx \\
add ax, -val4\\

\section{Problem 11}
what is the property of returned value of SIZEOF operator?\\
The product of  the returned value of LENGTHOF and the returned value of SIZE.

\section{Problem 12}
Which 32 bit general register can be used as indirect operand?\\
EAX, EBX, ECX, EDX, ESI, EDI, EBP, ESP

\section{Problem 13}
In real-address mode, which register is regarded as counter by LOOP operator? \\
CX register

\section{Problem 14}
What is little endian order? \\
Little endian order refers to the way Intel stores integers in memory. Multi-byte integers are stored in reverse order, with the least significant byte stored at the lowest address.

\section{Problem 15}
Does INC instruction affect CF? \\
No, it doesn't.

\end{document}
