%%%%%%%%%%%%%%%%%%%%%%%%%%%%%%%%%%%%%%%%%
% Short Sectioned Assignment
% LaTeX Template
% Version 1.0 (5/5/12)
%
% This template has been downloaded from:
% http://www.LaTeXTemplates.com
%
% Original author:
% Frits Wenneker (http://www.howtotex.com)
%
% License:
% CC BY-NC-SA 3.0 (http://creativecommons.org/licenses/by-nc-sa/3.0/)
%
%%%%%%%%%%%%%%%%%%%%%%%%%%%%%%%%%%%%%%%%%

%----------------------------------------------------------------------------------------
%	PACKAGES AND OTHER DOCUMENT CONFIGURATIONS
%----------------------------------------------------------------------------------------

\documentclass[paper=a4, fontsize=11pt]{scrartcl} % A4 paper and 11pt font size

\usepackage[T1]{fontenc} % Use 8-bit encoding that has 256 glyphs
\usepackage{fourier} % Use the Adobe Utopia font for the document - comment this line to return to the LaTeX default
\usepackage[english]{babel} % English language/hyphenation
\usepackage{amsmath,amsfonts,amsthm} % Math packages
\usepackage[CJKbookmarks=true,
            colorlinks,linkcolor=black,anchorcolor=blue,citecolor=green]{hyperref}

\usepackage{lipsum} % Used for inserting dummy 'Lorem ipsum' text into the template

\usepackage{sectsty} % Allows customizing section commands
\allsectionsfont{\centering \normalfont\scshape} % Make all sections centered, the default font and small caps

\usepackage{fancyhdr} % Custom headers and footers
\pagestyle{fancyplain} % Makes all pages in the document conform to the custom headers and footers
\fancyhead{} % No page header - if you want one, create it in the same way as the footers below
\fancyfoot[L]{} % Empty left footer
\fancyfoot[C]{} % Empty center footer
\fancyfoot[R]{\thepage} % Page numbering for right footer
\renewcommand{\headrulewidth}{0pt} % Remove header underlines
\renewcommand{\footrulewidth}{0pt} % Remove footer underlines
\renewcommand\thesection{\roman{section}}

\setlength{\headheight}{13.6pt} % Customize the height of the header

\numberwithin{equation}{section} % Number equations within sections (i.e. 1.1, 1.2, 2.1, 2.2 instead of 1, 2, 3, 4)
\numberwithin{figure}{section} % Number figures within sections (i.e. 1.1, 1.2, 2.1, 2.2 instead of 1, 2, 3, 4)
\numberwithin{table}{section} % Number tables within sections (i.e. 1.1, 1.2, 2.1, 2.2 instead of 1, 2, 3, 4)

\setlength\parindent{0pt} % Removes all indentation from paragraphs - comment this line for an assignment with lots of text


\usepackage{listings}
\usepackage{color}

\definecolor{dkgreen}{rgb}{0,0.6,0}
\definecolor{gray}{rgb}{0.5,0.5,0.5}
\definecolor{mauve}{rgb}{0.58,0,0.82}

\lstset{ %
    basicstyle=\small,%
    escapeinside=``,%
    keywordstyle=\color{red} \bfseries,% \underbar,%
    identifierstyle={},%
    commentstyle=\color{blue},%
    stringstyle=\ttfamily,%
    %labelstyle=\tiny,%
    extendedchars=false,%
    linewidth=\textwidth,%
    numbers=left,%
    numberstyle=\tiny \color{blue},%
    frame=trbl%
}




%----------------------------------------------------------------------------------------
%	TITLE SECTION
%----------------------------------------------------------------------------------------

\newcommand{\horrule}[1]{\rule{\linewidth}{#1}} % Create horizontal rule command with 1 argument of height

\title{	
\normalfont \normalsize
\textsc{School of Software, Tsinghua University} \\ [25pt] % Your university, school and/or department name(s)
\horrule{0.5pt} \\[0.4cm] % Thin top horizontal rule
\huge Architecture of Computer and Network (2)\\ % The assignment title
\LARGE\textit{homework 2}
\horrule{2pt} \\[0.5cm] % Thick bottom horizontal rule
}

\author{Qingfu Wen \\ \normalsize 2011013239} % Your Info
\date{\normalsize\today} % Today's date or a custom date

\begin{document}

\maketitle % Print the title
\tableofcontents
\newpage
%----------------------------------------------------------------------------------------
%	PROBLEM 1
%----------------------------------------------------------------------------------------

\section{Problem 1}
Which instruction pushes all 32 bits general register into stack?\\
PUSHAD

\section{Problem 2}
In protect mode,how many bytes of space does procedure need at least? \\
4 bytes


\section{Problem 3}
Please write a instruction to reverse all the bit in EAX.\\
xor eax, FFFFFFFFh

\section{Problem 4}
Which jump instruction equals to JA? \\
JNBE

\section*{Problem 5}
Please write some instructions to jump to label L2 when signed integer in AX is larger than signed integer in CX.\\
cmp ax, cx \\
jg L2

\section{Problem 6}
Does LOOPZ instruction jump to the label only when ZF = 0?\\
No, it jumps when ZF = 0 or ECX $\leq$ 0.

\section{Problem 7}
Use assembly language to implement the following pseudocode(assume that they are all unsigned)\\
\begin{lstlisting}[language={[ANSI]C}]
if (dx <= cx)
     X = 1;
else
     X = 2;
\end{lstlisting}


\begin{lstlisting}[language={[ANSI]C}]
cmp dx, cx
jbe L1
jmp L2
L1:
  mov x, 1
L2:
  mov x, 2
\end{lstlisting}


\section{Problem 8}
Use 32 bits register to implement the following loop(do not use directives like .while)\\
\begin{lstlisting}[language={[ANSI]C}]
while( ebx <= val1)
{
	ebx = ebx + 5;
	val1 = val1 - 1
}
\end{lstlisting}
\begin{lstlisting}[language={[ANSI]C}]
while:
    cmp ebx, val1
    jg endwhile
    add ebx, 5
    dec val1
    jmp while
endwhile
\end{lstlisting}

\end{document}
