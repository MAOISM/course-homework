%%%%%%%%%%%%%%%%%%%%%%%%%%%%%%%%%%%%%%%%%
% Short Sectioned Assignment
% LaTeX Template
% Version 1.0 (5/5/12)
%
% This template has been downloaded from:
% http://www.LaTeXTemplates.com
%
% Original author:
% Frits Wenneker (http://www.howtotex.com)
%
% License:
% CC BY-NC-SA 3.0 (http://creativecommons.org/licenses/by-nc-sa/3.0/)
%
%%%%%%%%%%%%%%%%%%%%%%%%%%%%%%%%%%%%%%%%%

%----------------------------------------------------------------------------------------
%	PACKAGES AND OTHER DOCUMENT CONFIGURATIONS
%----------------------------------------------------------------------------------------

\documentclass[paper=a4, fontsize=11pt]{scrartcl} % A4 paper and 11pt font size

\usepackage[T1]{fontenc} % Use 8-bit encoding that has 256 glyphs
\usepackage{fourier} % Use the Adobe Utopia font for the document - comment this line to return to the LaTeX default
\usepackage[english]{babel} % English language/hyphenation
\usepackage{amsmath,amsfonts,amsthm} % Math packages
\usepackage{pgfplots, tikz}
\usetikzlibrary{intersections}
\usepackage[CJKbookmarks=true,
            colorlinks,linkcolor=black,anchorcolor=blue,citecolor=green]{hyperref}

\usepackage{lipsum} % Used for inserting dummy 'Lorem ipsum' text into the template

\usepackage{sectsty} % Allows customizing section commands
\allsectionsfont{\centering \normalfont\scshape} % Make all sections centered, the default font and small caps

\usepackage{fancyhdr} % Custom headers and footers
\pagestyle{fancyplain} % Makes all pages in the document conform to the custom headers and footers
\fancyhead{} % No page header - if you want one, create it in the same way as the footers below
\fancyfoot[L]{} % Empty left footer
\fancyfoot[C]{} % Empty center footer
\fancyfoot[R]{\thepage} % Page numbering for right footer
\renewcommand{\headrulewidth}{0pt} % Remove header underlines
\renewcommand{\footrulewidth}{0pt} % Remove footer underlines
\renewcommand\thesection{\roman{section}}

\setlength{\headheight}{13.6pt} % Customize the height of the header

\numberwithin{equation}{section} % Number equations within sections (i.e. 1.1, 1.2, 2.1, 2.2 instead of 1, 2, 3, 4)
\numberwithin{figure}{section} % Number figures within sections (i.e. 1.1, 1.2, 2.1, 2.2 instead of 1, 2, 3, 4)
\numberwithin{table}{section} % Number tables within sections (i.e. 1.1, 1.2, 2.1, 2.2 instead of 1, 2, 3, 4)

\setlength\parindent{0pt} % Removes all indentation from paragraphs - comment this line for an assignment with lots of text

%----------------------------------------------------------------------------------------
%	TITLE SECTION
%----------------------------------------------------------------------------------------

\newcommand{\horrule}[1]{\rule{\linewidth}{#1}} % Create horizontal rule command with 1 argument of height

\title{	
\normalfont \normalsize
\textsc{School of Software, Tsinghua University} \\ [25pt] % Your university, school and/or department name(s)
\horrule{0.5pt} \\[0.4cm] % Thin top horizontal rule
\huge Optimization Method\\ % The assignment title
\LARGE\textit{homework 5}
\horrule{2pt} \\[0.5cm] % Thick bottom horizontal rule
}

\author{Qingfu Wen \\ \normalsize 2015213495} % Your Info
\date{\normalsize\today} % Today's date or a custom date

\begin{document}

\maketitle % Print the title
\tableofcontents
\newpage
%----------------------------------------------------------------------------------------
%	PROBLEM 1
%----------------------------------------------------------------------------------------
\section{Problem 1}
Suppose the original problem is 
\begin{alignat}{2}          \nonumber
\min\quad & 4x_1+3x_2+x_3 \\    \nonumber
\mbox{s.t.}\quad            \nonumber
& x_1-x_2+x_3 \geq 1\\        \nonumber
& x_1+2x_2-3x_3 \geq 2\\         \nonumber
& x_1,x_2,x_3 \geq 0
\end{alignat}
the optimized solution of the dural problem is $(w_1,w_2)=(\frac{5}{3},\frac{7}{3})$, solve the original problem using dual property.
\emph{\textbf{Solution:}}\\
Dural problem:
\begin{alignat}{2}          \nonumber
\max\quad & w_1+2w_2 \\    \nonumber
\mbox{s.t.}\quad            \nonumber
& w_1+w_2 \leq 4\\        \nonumber
& -w_1+2w_2\leq 3\\         \nonumber
& w_1-3w_2\leq 1\\         \nonumber
& w_1,w_2 \geq 0
\end{alignat}
Since the optimized solution of the dural problem is $w_1>0$, $w_2>0$, from the complementary slackness theorem and the constraint on the third  inequality in the dural problem is tight, we get
\begin{equation}
\left\{
\begin{aligned} \nonumber
x_1-x_2+x_3 &= 1 \\
x_1+2x_2-3x_3 &=2 \\
x_3 &= 0 \\
\end{aligned}
\right.
\end{equation}
so the optimized solution of the original problem is $x_1=\frac{4}{3}$, $x_2=\frac{1}{3}$, $x_3=0$, $f_{min}=\frac{19}{3}$
\section{Problem 2}
Given a linear programming problem:
\begin{alignat}{2}          \nonumber
\min\quad & 5x_1+21x_3 \\    \nonumber
\mbox{s.t.}\quad            \nonumber
& x_1-x_2+6x_3 \geq b_1\\        \nonumber
& x_1+x_2+2x_3 \geq 1\\         \nonumber
& x_1,x_2,x_3 \geq 0
\end{alignat}
$b_1$is a positive number, and the optimized solution of the problem is $(x_1, x_2, x_3)=(\frac{1}{2}, 0, \frac{1}{4})$.
\begin{enumerate}
\item write out the dural problem.\\
\begin{alignat}{2}          \nonumber
\max\quad & b_1w_1+w_2 \\    \nonumber
\mbox{s.t.}\quad & w_1+w_2 \leq 5\\        \nonumber
& -w_1+w_2\leq 0\\         \nonumber
& 6w_1+2w_2\leq 21\\         \nonumber
& w_1,w_2 \geq 0
\end{alignat}
\item get the optimized solution of the dural problem.\\
using the complementary slackness theorem,since $w_1>0$, from $x_1-x_2+6x_3=b_1$ we get $b_1=2$.And the optimized solution $x_1>0$,$x_3>0$, we get
\begin{equation}
\left\{
\begin{aligned} \nonumber
w_1+w_2 &= 5 \\
6w_1+2w_2 &=21 \\
\end{aligned}
\right.
\end{equation}
Then, we get the optimized solution $w_1=\frac{11}{4}$, $w_2=\frac{9}{4}$,$f_{min}=\frac{31}{4}$
\end{enumerate}

\section{Problem 3}
Considering the linear problem:
\begin{alignat}{2}          \nonumber
\min\quad & cx \\    \nonumber
\mbox{s.t.}\quad            \nonumber
& Ax=b\\        \nonumber
& x \geq 0
\end{alignat}
$A$ is a $m*m$ symmetrical matrix, $c^T=b$. Prove that if $x^{(0)}$ is a feasible solution, it is optimized.\\
\emph{\textbf{Proof:}}\\
Dural problem:
\begin{alignat}{2}          \nonumber
\max\quad & wb \\    \nonumber
\mbox{s.t.}\quad & wA \leq c\\        \nonumber
\end{alignat}
we can know that $w={x^{(0)}}^T$ is a feasible solution and $c{x^(0)}=w^{(0)}b$, so $x^{(0)}$is optimized.
\section{Problem 4}
solve the following LP problem using dual simplex method.
\begin{enumerate}
\item
\begin{alignat}{2}          \nonumber
\max\quad & x_1+x_2 \\    \nonumber
\mbox{s.t.}\quad            \nonumber
& x_1-x_2-x_3 = 1\\        \nonumber
& -x_1+x_2+2x_3 \geq 1\\         \nonumber
& x_1,x_2,x_3 \geq 0
\end{alignat}
\emph{\textbf{Solution:}}\\
the extended problem:\\
\begin{alignat}{2}          \nonumber
\max\quad & x_1+x_2 \\    \nonumber
\mbox{s.t.}\quad            \nonumber
& x_1-x_2-x_3 = 1\\        \nonumber
& -x_3+x_4= -2\\         \nonumber
& x_2+x_3+x_5 = M\\          \nonumber
& x_j \geq 0, j = 1,\cdots,5
\end{alignat}

\begin{tabular}{|c|c|c|c|c|c|c|c|c|}
\hline &$x_1$&$x_2$&$x_3$&$x_4$&$x_5$&\\
\hline$x_1$&1&-1&-1&0&0&1\\
$x_4$&0&0&-1&1&0&-2\\
$x_5$&0&\Large{\textcircled{\small{1}}}&1&0&1&M\\
\hline f&0&-2&-1&0&0&1\\
\hline
\hline $x_1$&1& 0 &0& 0&1&M+1\\
$x_4$&0&0&\Large{\textcircled{\small{-1}}}&1&0&-2\\
$x_2$&0&1&1&0&1&M\\
\hline f & 0 & 0 & 1 & 0 &2 &2M+1\\
\hline
\hline $x_1$&1& 0 &0& 0&1&M+1\\
$x_3$&0&0&1&-1&0&2\\
$x_2$&0&1&0&1&1&M-2\\
\hline f & 0 & 0 & 0 & 1 &2 &2M-1\\
\hline
\end{tabular}
\\\\
From the above table, we can see that the optimized solution is $(M+1,M-2,2,0,0)$,$f_{max}=2M-1$. Since $M\rightarrow\infty$, the problem has no upper bound.


\item
\begin{alignat}{2}          \nonumber
\min\quad & 4x_1+3x_2+5x_3+x_4+2x_5\\    \nonumber
\mbox{s.t.}\quad            \nonumber
& -x_1+2x_2-2x_3+3x_4-3x_5+x_6+x_8 = 1\\        \nonumber
& x_1+x_2-3x_3+2x_4-2x_5+x_8 = 4\\         \nonumber
& -2x_3+3x_4-3x_5+x_7+x_8 = 2\\         \nonumber
& x_j\geq0,j=1,\cdots,8
\end{alignat}

\emph{\textbf{Solution:}}\\

\begin{tabular}{|c|c|c|c|c|c|c|c|c|c|c|}
\hline &$x_1$&$x_2$&$x_3$&$x_4$&$x_5$&$x_6$&$x_7$&$x_8$&\\
\hline$x_6$&-2&1&1&1&\Large{\textcircled{\small{-1}}}&1&0&0&-3\\
\hline$x_8$&1&1&-3&2&-2&0&0&1&4\\
\hline$x_7$&-1&-1&1&1&-1&0&1&0&-2\\
\hline f&-4&-3&-5&-1&-2&0&0&0&0\\
\hline
\hline$x_5$&2&-1&-1&-1&1&-1&0&0&3\\
\hline$x_8$&5&-1&-5&0&0&-2&0&1&10\\
\hline$x_7$&1&-2&0&0&0&-1&1&0&1\\
\hline f&0&-5&-7&-3&0&-2&0&0&6\\
\hline
\end{tabular}
\\\\
From the above table, we can get the optimized solution $\mathbf{x}$=(0,0,0,0,3,0,1,10),$f_{min}=6$.

\end{enumerate}

\end{document}
