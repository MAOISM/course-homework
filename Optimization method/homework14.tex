%%%%%%%%%%%%%%%%%%%%%%%%%%%%%%%%%%%%%%%%%
% Short Sectioned Assignment
% LaTeX Template
% Version 1.0 (5/5/12)
%
% This template has been downloaded from:
% http://www.LaTeXTemplates.com
%
% Original author:
% Frits Wenneker (http://www.howtotex.com)
%
% License:
% CC BY-NC-SA 3.0 (http://creativecommons.org/licenses/by-nc-sa/3.0/)
%
%%%%%%%%%%%%%%%%%%%%%%%%%%%%%%%%%%%%%%%%%

%----------------------------------------------------------------------------------------
%	PACKAGES AND OTHER DOCUMENT CONFIGURATIONS
%----------------------------------------------------------------------------------------

\documentclass[paper=a4, fontsize=11pt]{scrartcl} % A4 paper and 11pt font size

\usepackage[T1]{fontenc} % Use 8-bit encoding that has 256 glyphs
\usepackage{fourier} % Use the Adobe Utopia font for the document - comment this line to return to the LaTeX default
\usepackage[english]{babel} % English language/hyphenation
\usepackage{amsmath,amsfonts,amsthm} % Math packages
\usepackage{pgfplots, tikz,comment}
\usetikzlibrary{intersections}
\usepackage[CJKbookmarks=true,
            colorlinks,linkcolor=black,anchorcolor=blue,citecolor=green]{hyperref}

\usepackage{lipsum} % Used for inserting dummy 'Lorem ipsum' text into the template

\usepackage{sectsty} % Allows customizing section commands
\allsectionsfont{\centering \normalfont\scshape} % Make all sections centered, the default font and small caps

\usepackage{fancyhdr} % Custom headers and footers
\pagestyle{fancyplain} % Makes all pages in the document conform to the custom headers and footers
\fancyhead{} % No page header - if you want one, create it in the same way as the footers below
\fancyfoot[L]{} % Empty left footer
\fancyfoot[C]{} % Empty center footer
\fancyfoot[R]{\thepage} % Page numbering for right footer
\renewcommand{\headrulewidth}{0pt} % Remove header underlines
\renewcommand{\footrulewidth}{0pt} % Remove footer underlines
\renewcommand\thesection{\roman{section}}

\setlength{\headheight}{13.6pt} % Customize the height of the header

\numberwithin{equation}{section} % Number equations within sections (i.e. 1.1, 1.2, 2.1, 2.2 instead of 1, 2, 3, 4)
\numberwithin{figure}{section} % Number figures within sections (i.e. 1.1, 1.2, 2.1, 2.2 instead of 1, 2, 3, 4)
\numberwithin{table}{section} % Number tables within sections (i.e. 1.1, 1.2, 2.1, 2.2 instead of 1, 2, 3, 4)

\setlength\parindent{0pt} % Removes all indentation from paragraphs - comment this line for an assignment with lots of text

%----------------------------------------------------------------------------------------
%	TITLE SECTION
%----------------------------------------------------------------------------------------

\newcommand{\horrule}[1]{\rule{\linewidth}{#1}} % Create horizontal rule command with 1 argument of height

\title{	
\normalfont \normalsize
\textsc{School of Software, Tsinghua University} \\ [25pt] % Your university, school and/or department name(s)
\horrule{0.5pt} \\[0.4cm] % Thin top horizontal rule
\huge Optimization Method\\ % The assignment title
\LARGE\textit{homework 14}
\horrule{2pt} \\[0.5cm] % Thick bottom horizontal rule
}

\author{Qingfu Wen \\ \normalsize 2015213495} % Your Info
\date{\normalsize\today} % Today's date or a custom date

\begin{document}

\maketitle % Print the title
\tableofcontents
\newpage
%----------------------------------------------------------------------------------------
%	PROBLEM 1
%----------------------------------------------------------------------------------------
\section{Problem 1}
Consider the following problem:
\begin{alignat}{2}          \nonumber
\min\quad & x_1^2+x_1x_2+2x_2^2-6x_1-2x_2-12x_3 \\    \nonumber
\mbox{s.t.}\quad            \nonumber
& x_1+x_2+x_3 = 2 \\    \nonumber
& -x_1+2x_2 \leq 3 \\     \nonumber
& x_1,x_2,x_3 \geq 0       \nonumber
\end{alignat}\\
Please compute the descent feasible direction at point $\hat{x}=(1,1,0)^T$.
\\~\\
\emph{\textbf{Solution:}}\\
~\\
$\hat{x}$'s active constraint are $x_1+x_2+x_3=0$ and $x_3\geq0$. so feasible direction $d$ holds
\begin{equation}
\left\{
\begin{aligned}
d_1+d_2+d_3 =0\\
d_3\geq0\\
\end{aligned}
\right. \nonumber
\end{equation}
As a descent direction, we have $\nabla f(\hat{x})d<0$. Since\\
\begin{equation} \nonumber
\nabla f(\hat{x}) = \begin{bmatrix}2x_1+x_2-6\\x_1+4x_2-2\\-12\end{bmatrix} = \begin{bmatrix}-3\\3\\-12\end{bmatrix}
\end{equation}
So the descent feasible direction holds 
\begin{equation}
\left\{
\begin{aligned}
d_1+d_2+d_3 =0\\
d_3\geq0\\
-3d_1+3d_2-12d_3 < 0
\end{aligned}
\right. \nonumber
\end{equation}
A possible solution is $d=(-1,0,1)$.

\section{Problem 2}
Consider the following problem:
\begin{alignat}{2}          \nonumber
\min\quad & f(x) \\    \nonumber
\mbox{s.t.}\quad            \nonumber
& g_i(x) \geq 0, i=1,2,\cdots,m \\    \nonumber
& h_j(x) =0, j=1,2,\cdots,l\\     \nonumber
\end{alignat}\\
Suppose $\hat{x}$ is a feasible point, $I=\{i|g_i(\hat{x})=0\}$.\\
Please prove $\hat{x}$ is a KTT point's Necessary and Sufficient Condition is that the optimum of the following objective function is 0:
\begin{alignat}{2}          \nonumber
\min\quad & \nabla f(\hat{x})^Td \\    \nonumber
\mbox{s.t.}\quad            \nonumber
& \nabla g_i(\hat{x})^Td \geq0, i\in I \\    \nonumber
& \nabla h_j(\hat{x})^Td =0, j=1,2,\cdots,l\\ \nonumber
& -1 \leq d_j \leq 1, j =1,2,\cdots,n
\end{alignat}\\
\emph{\textbf{Proof:}}\\~\\
$\hat{x}$ is a KTT point whose Necessary and Sufficient Condition is that $\exists w_i\geq0(i\in I)$ and $v_j$,
\begin{equation} \label{equ 2.1}
\nabla f(\hat{x}) - \sum_{i\in I}w_i\nabla g_i(\hat{x})-\sum_{j=1}^{l}v_j\nabla h_j(\hat{x}) = 0
\end{equation}
Mark $A_1$ = $\left[\nabla g_{i1}(\hat{x}),\cdots,\nabla g_{im}(\hat{x})\right]$, $E=\left[\nabla h_1(\hat{x},\cdots,\nabla h_l(\hat{x}\right]$, $v = p-q(p\geq0,q\geq0)$. We can rewrite equation \ref{equ 2.1} into
\begin{equation} \label{equ 2.2}
(-A_1, -E, E)\begin{bmatrix}w\\p\\q\end{bmatrix}=-\nabla f(\hat{x}), \begin{bmatrix}w\\p\\q\end{bmatrix}\geq 0
\end{equation}
From Farkas Theorem, we know \ref{equ 2.2} has solution whose Necessary and Sufficient Condition is
\begin{equation}
\begin{bmatrix}-A_1^T\\ -E^T\\ E^T\end{bmatrix}d\leq0, -\nabla f(\hat{x})^Td> 0
\end{equation}
has no solution. That is
\begin{equation}
\left\{
\begin{aligned}
\nabla f(\hat{x})^Td< 0\\
A_1^Td\geq0\\
E^Td = 0
\end{aligned}
\right. \nonumber
\end{equation}
has no solution. Thus,
\begin{alignat}{2}          \nonumber
\min\quad & \nabla f(\hat{x})^Td \\    \nonumber
\mbox{s.t.}\quad            \nonumber
& \nabla g_i(\hat{x})^Td \geq0, i\in I \\    \nonumber
& \nabla h_j(\hat{x})^Td =0, j=1,2,\cdots,l\\ \nonumber
& -1 \leq d_j \leq 1, j =1,2,\cdots,n
\end{alignat}
has optimal solution 0.
\end{document}
