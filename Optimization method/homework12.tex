%%%%%%%%%%%%%%%%%%%%%%%%%%%%%%%%%%%%%%%%%
% Short Sectioned Assignment
% LaTeX Template
% Version 1.0 (5/5/12)
%
% This template has been downloaded from:
% http://www.LaTeXTemplates.com
%
% Original author:
% Frits Wenneker (http://www.howtotex.com)
%
% License:
% CC BY-NC-SA 3.0 (http://creativecommons.org/licenses/by-nc-sa/3.0/)
%
%%%%%%%%%%%%%%%%%%%%%%%%%%%%%%%%%%%%%%%%%

%----------------------------------------------------------------------------------------
%	PACKAGES AND OTHER DOCUMENT CONFIGURATIONS
%----------------------------------------------------------------------------------------

\documentclass[paper=a4, fontsize=11pt]{scrartcl} % A4 paper and 11pt font size

\usepackage[T1]{fontenc} % Use 8-bit encoding that has 256 glyphs
\usepackage{fourier} % Use the Adobe Utopia font for the document - comment this line to return to the LaTeX default
\usepackage[english]{babel} % English language/hyphenation
\usepackage{amsmath,amsfonts,amsthm} % Math packages
\usepackage{pgfplots, tikz,comment}
\usetikzlibrary{intersections}
\usepackage[CJKbookmarks=true,
            colorlinks,linkcolor=black,anchorcolor=blue,citecolor=green]{hyperref}

\usepackage{lipsum} % Used for inserting dummy 'Lorem ipsum' text into the template

\usepackage{sectsty} % Allows customizing section commands
\allsectionsfont{\centering \normalfont\scshape} % Make all sections centered, the default font and small caps

\usepackage{fancyhdr} % Custom headers and footers
\pagestyle{fancyplain} % Makes all pages in the document conform to the custom headers and footers
\fancyhead{} % No page header - if you want one, create it in the same way as the footers below
\fancyfoot[L]{} % Empty left footer
\fancyfoot[C]{} % Empty center footer
\fancyfoot[R]{\thepage} % Page numbering for right footer
\renewcommand{\headrulewidth}{0pt} % Remove header underlines
\renewcommand{\footrulewidth}{0pt} % Remove footer underlines
\renewcommand\thesection{\roman{section}}

\setlength{\headheight}{13.6pt} % Customize the height of the header

\numberwithin{equation}{section} % Number equations within sections (i.e. 1.1, 1.2, 2.1, 2.2 instead of 1, 2, 3, 4)
\numberwithin{figure}{section} % Number figures within sections (i.e. 1.1, 1.2, 2.1, 2.2 instead of 1, 2, 3, 4)
\numberwithin{table}{section} % Number tables within sections (i.e. 1.1, 1.2, 2.1, 2.2 instead of 1, 2, 3, 4)

\setlength\parindent{0pt} % Removes all indentation from paragraphs - comment this line for an assignment with lots of text

%----------------------------------------------------------------------------------------
%	TITLE SECTION
%----------------------------------------------------------------------------------------

\newcommand{\horrule}[1]{\rule{\linewidth}{#1}} % Create horizontal rule command with 1 argument of height

\title{	
\normalfont \normalsize
\textsc{School of Software, Tsinghua University} \\ [25pt] % Your university, school and/or department name(s)
\horrule{0.5pt} \\[0.4cm] % Thin top horizontal rule
\huge Optimization Method\\ % The assignment title
\LARGE\textit{homework 12}
\horrule{2pt} \\[0.5cm] % Thick bottom horizontal rule
}

\author{Qingfu Wen \\ \normalsize 2015213495} % Your Info
\date{\normalsize\today} % Today's date or a custom date

\begin{document}

\maketitle % Print the title
\tableofcontents
\newpage
%----------------------------------------------------------------------------------------
%	PROBLEM 1
%----------------------------------------------------------------------------------------
\section{Problem 1}
For the following non-linear programming problem.
\begin{alignat}{2}          \nonumber
\min\quad & x_2\\    \nonumber
\mbox{s.t.}\quad            \nonumber
& -x_1^2-(x_2-4)^2+16 \geq 0\\        \nonumber
& (x_1-2)^2+(x_2-3)^2-13 = 0\\         \nonumber
\end{alignat}
Judge whether they are local optimal solution or not
\begin{equation}
x^{(1)}=\begin{pmatrix} 0 \\ 0\end{pmatrix}, \quad
x^{(2)}=\begin{pmatrix} \frac{16}{5} \\ \frac{32}{5} \end{pmatrix}, \quad
x^{(3)}=\begin{pmatrix} 2 \\ 3+\sqrt{13}\end{pmatrix} \nonumber
\end{equation}
\emph{\textbf{Solution:}}
$f(x)=x_2$, $g(x)=-x_1^2-(x_2-4)^2+16$, $h(x)=(x_1-2)^2+(x_2-3)^2-13$\\
\begin{equation} \nonumber
\nabla f(x) = \begin{bmatrix}0 \\ 1\end{bmatrix}
\nabla g(x) = \begin{bmatrix}-2x_1 \\ -2(x_2-4)\end{bmatrix}
\nabla h(x) = \begin{bmatrix}2(x_1-2) \\ 2(x_2-3)\end{bmatrix}
\end{equation}
Lagrange function $L(x,w,v)=x_2-w[-x_1^2-(x_2-4)^2+16]-v[(x_1-2)^2+(x_2-3)^2-13]$\\
$\nabla^2_xL(x,w,v)=\begin{bmatrix}2(w-v) &0\\0& 2(w-v)\end{bmatrix}$\\
\\
For $x^{(1)}=\begin{pmatrix} 0 \\ 0\end{pmatrix}$
\begin{equation} \nonumber
\nabla f(x) = \begin{bmatrix}0 \\ 1\end{bmatrix}
\nabla g(x) = \begin{bmatrix}0 \\ 8\end{bmatrix}
\nabla h(x) = \begin{bmatrix}-4 \\ -6\end{bmatrix}
\end{equation}
K-T-T condition is
\begin{equation} \nonumber
\left\{
\begin{aligned}
4v &= 0\\
1-8w+6v &= 0\\
w \geq 0
\end{aligned}
\right.
\end{equation}
when $w=\frac{1}{8}$, $v=0$, the K-T-T condition holds.\\
\begin{equation} \nonumber
\left\{
\begin{aligned}
\nabla g(x^{(1)})^Td &= 0\\
\nabla h(x^{(1)})^Td &= 0\\
\end{aligned}
\right.
\end{equation}
we get $d=0$ and $G=\emptyset$, so $x^{(1)}$ is an optimal solution.\\
\\
For $x^{(2)}=\begin{pmatrix} \frac{16}{5} \\ \frac{32}{5}\end{pmatrix}$
\begin{equation} \nonumber
\nabla f(x) = \begin{bmatrix}0 \\ 1\end{bmatrix}
\nabla g(x) = \begin{bmatrix}-\frac{32}{5}\\ -\frac{24}{5}\end{bmatrix}
\nabla h(x) = \begin{bmatrix}\frac{12}{5}\\ \frac{34}{5}\end{bmatrix}
\end{equation}
K-T-T condition is
\begin{equation} \nonumber
\left\{
\begin{aligned}
\frac{32}{5}w-\frac{12}{5}v &= 0\\
1+\frac{24}{5}w-\frac{34}{5}v &= 0\\
w \geq 0
\end{aligned}
\right.
\end{equation}
when $w=\frac{3}{40}$, $v=\frac{1}{5}$, the K-T-T condition holds.\\
\begin{equation} \nonumber
\left\{
\begin{aligned}
\nabla g(x^{(2)})^Td &= 0\\
\nabla h(x^{(2)})^Td &= 0\\
\end{aligned}
\right.
\end{equation}
we get $d=0$ and $G=\emptyset$, so $x^{(2)}$ is an optimal solution.\\
\\
For $x^{(3)}=\begin{pmatrix} 2 \\ 3+\sqrt{13}\end{pmatrix}$
\begin{equation} \nonumber
\nabla f(x) = \begin{bmatrix}0\\1\end{bmatrix}
\nabla h(x) = \begin{bmatrix}0\\2\sqrt{13}\end{bmatrix}
\end{equation}
K-T-T condition is
\begin{equation} \nonumber
\left\{
\begin{aligned}
1-2\sqrt{13}v &= 0\\
\end{aligned}
\right.
\end{equation}
when $v=\frac{\sqrt{13}}{26}$, the K-T-T condition holds.\\
\begin{equation} \nonumber
\left\{
\begin{aligned}
\nabla h(x^{(3)})^Td &= 0\\
\end{aligned}
\right.
\end{equation}
we get $d=\begin{bmatrix}d_1\\0\end{bmatrix},d_1\neq0$ and $G=\{d|d=\begin{bmatrix}d_1\\0\end{bmatrix},d_1\neq0\}$.
\begin{equation}\nonumber
\nabla^2_xL(x^{(3)},w,v) = \begin{bmatrix}-\frac{1}{\sqrt{13}}& 0\\0 & -\frac{1}{\sqrt{13}}\end{bmatrix}
\end{equation}
\begin{equation}\nonumber
d^T\nabla^2_xL(x^{(3)},w,v)d =(d_1,0)\begin{bmatrix}-\frac{1}{\sqrt{13}}& 0\\0 & -\frac{1}{\sqrt{13}}\end{bmatrix}
\begin{bmatrix}d_1\\0\end{bmatrix}=-\frac{1}{\sqrt{13}}d^2_1 <0
\end{equation}
so $x^{(3)}$ is not an optimal solution.
\section{Problem 2}
Given a non-linear programming problem.
\begin{alignat}{2}          \nonumber
\max\quad & b^Tx \quad  x\in R^n \\    \nonumber
\mbox{s.t.}\quad            \nonumber
& xx^T \leq 1\\         \nonumber
\end{alignat}
$b \neq 0$, prove that $\overline{x} = \frac{b}{||b||}$ holds on necessary condition of optimality.\\
\\
\emph{\textbf{Proof:}}\\
we can rewrite the non-linear programming problem into:
\begin{alignat}{2}          \nonumber
\min\quad & -b^Tx \quad  x\in R^n \\    \nonumber
\mbox{s.t.}\quad            \nonumber
& 1-xx^T\geq 0\\         \nonumber
\end{alignat}
From K-K-T condition, we can know that:
\begin{equation} \nonumber
\left\{
\begin{aligned}
-b+\omega x &= 0\\
\omega(1-x^Tx) &= 0\\
\omega \geq 0
\end{aligned}
\right.
\end{equation}
we can get K-T-T point $x=\frac{b}{||b||}$. Besides, we know the above problem is a convex programming problem, K-T-T condition is the sufficient condition of optimal solution.
\section{Problem 3}
Given a non-linear programming problem.
\begin{alignat}{2}          \nonumber
\min\quad & \frac{1}{2}[(x_1-1)^2+x_2^2]\\    \nonumber
\mbox{s.t.}\quad            \nonumber
& -x_1+\beta x_2^2 = 0\\         \nonumber
\end{alignat}
Discuss $\beta$'s span when $\overline{x}=(0,0)^T$ becomes local optimal solution.\\
\\
\emph{\textbf{Solution:}}\\
we can know $f(x)= \frac{1}{2}[(x_1-1)^2+x_2^2]$, $h(x)=-x_1+\beta x_2^2$,\\
$\nabla f(x)= \begin{bmatrix} x_1-1 \\ x_2\end{bmatrix}$ , $\nabla h(x) = \begin{bmatrix} -1 \\ 2\beta x_2\end{bmatrix}$, from K-T-T condition, we have\\
\begin{equation} \nonumber
\left\{
\begin{aligned}
x_1-1+v &= 0\\
x_2-2\beta vx_2 &= 0\\
\omega \geq 0
\end{aligned}
\right.
\end{equation}
Since $\overline{x}=(0,0)^T$, we get $v=1$.
for Lagrange function
\begin{equation} \nonumber
L(x,v) = \frac{1}{2}[(x_1-1)^2+x_2^2] -v(-x_1+\beta x_2^2)
\end{equation}
at point $\overline{x}=(0,0)^T$,
\begin{equation} \nonumber
\nabla^2_x L(\overline{x},v) = \begin{bmatrix} 1 & 0\\0& 1-2\beta\end{bmatrix}, \nabla h(\overline{x}) = \begin{bmatrix}-1\\0\end{bmatrix},
\end{equation}
Direction set $\overline{G}=\{d|\nabla h(\overline(x))^Td=0\}=\{(0,d_2)^T|d_2\in \mathbb{R}\}$,
\begin{equation} \nonumber
(0,d_2)\begin{bmatrix} 1 & 0\\0& 1-2\beta\end{bmatrix}\begin{bmatrix} 0\\d_2\end{bmatrix} = (1-2\beta)d^2_2>0
\end{equation}
we get $\beta<\frac{1}{2}$, $\overline{x}$ is optimal solution. when $\beta=\frac{1}{2}$, original problem becomes non-constraint problem
\begin{equation} \nonumber
\min \frac{1}{2}(x_1^2+1)
\end{equation}
$\overline{x}$ is still optimal solution.\\
Above all, when $\beta\leq\frac{1}{2}$, $\overline{x}$ is optimal solution.
\section{Problem 4}
Given a non-linear programming problem.
\begin{alignat}{2}          \nonumber
\min\quad & (x_1-1)^2+(x_2+1)^2\\    \nonumber
\mbox{s.t.}\quad            \nonumber
& -x_1+x_2-1 \geq 0\\         \nonumber
\end{alignat}
\begin{enumerate}
\item   Solve this problem using graph method and optimality condition.
\\
\emph{\textbf{Solution:}}\\
we have $f(x)=(x_1-1)^2+(x_2+1)^2$, $g(x)=-x_1+x_2-1$,
$\nabla f(x)= \begin{bmatrix} 2(x_1-1) \\ 2(x_2+1)\end{bmatrix}$ , $\nabla g(x) = \begin{bmatrix}-1\\1\end{bmatrix}$
optimal condition is as follows:
\begin{equation} \nonumber
\left\{
\begin{aligned}
2(x_1-1)+w &= 0\\
2(x_2+1)-w &= 0\\
\omega(-x_1+x_2-1) &= 0\\
\omega &\geq 0\\
-x_1+x_2-1&\geq 0
\end{aligned}
\right.
\end{equation}
we know that when $x_1=-\frac{1}{2}$,$x_2=\frac{1}{2}$,$\omega=3$, optimal value $f_{min}=\frac{9}{2}$.
\item   Write out its dural problem.
\\
\emph{\textbf{Solution:}}\\
Lagrange function
\begin{equation} \nonumber
L(\omega) = (x_1-1)^2+(x_2+1)^2 - \omega(-x_1+x_2-1)
\end{equation}
Objective function of dural problem is
\begin{equation} \nonumber
\theta(\omega) = inf\{(x_1-1)^2+(x_2+1)^2 - \omega(-x_1+x_2-1) | x\in\mathbb{R}\}
\end{equation}
when $\omega\geq0$, $\theta(\omega)=-\frac{1}{2}\omega^2+3\omega$, the dural problem is
\begin{alignat}{2}          \nonumber
\max\quad & -\frac{1}{2}\omega^2+3\omega\\    \nonumber
\mbox{s.t.}\quad            \nonumber
&w \geq 0\\         \nonumber
\end{alignat}
\item   Solve the dural problem.
\begin{equation} \nonumber
\left\{
\begin{aligned}
-w+3+w_1 &= 0\\
ww_1 &= 0\\
w_1&\geq 0\\
w&\geq 0\\
\end{aligned}
\right.
\end{equation}
we get $w=3$,$w_1=0$, the optimal value is $\theta_{max}=\frac{9}{2}$
\end{enumerate}

\section{Problem 5}
\begin{alignat}{2}          \nonumber
\min\quad & c^Tx\\    \nonumber
\mbox{s.t.}\quad            \nonumber
& Ax = 0\\         \nonumber
& x^Tx \leq \gamma^2\\         \nonumber
\end{alignat}
$A$ is a $m*n (m<n)$ matrix, rank$(A) = m $, $c\in R^n$ and $c\neq0$, $\gamma$ is positive, What is the optimal solution and optimal value of objective function.\\
\\
\emph{\textbf{Proof:}}\\
Since $f(x)$ is a linear function, $g(x)$ is a concave function and $h(x)$ is a linear function, if the K-T-T condition holds at point $\overline{x}$, $\overline{x}$ is a optimal solution.\\
we can see that the feasible region is a convex set, $f_{min}$ exists at the edge of region. So we can rewrite the problem into
\begin{alignat}{2}          \nonumber
\min\quad & c^Tx\\    \nonumber
\mbox{s.t.}\quad            \nonumber
& Ax = 0\\         \nonumber
&  \gamma^2-x^Tx = 0\\         \nonumber
\end{alignat}
the K-T-T condition is
\begin{equation} \nonumber
\left\{
\begin{aligned}
c-A^Tv+2v_{m+1}x &= 0\\
Ax &= 0\\
\gamma^2-x^Tx &= 0\\
\end{aligned}
\right.
\end{equation}
$v=(v_1,v_2,\cdots,v_m)^T$ and $v_{m+1}$ is a K-T-T multiplier, then we can get the following result:\\
when $c\neq A^Tv$
\begin{equation} \nonumber
v = (AA^T)^{-1}Ac, v_{m+1}= -\frac{f_{min}}{2\gamma^2}\\
\end{equation}
\begin{equation} \nonumber
f_{min} = -r\sqrt{c^T(c-A^Tv)}
\end{equation}
\begin{equation} \nonumber
x = \frac{\gamma^2}{f_{min}}(c-A^Tv)
\end{equation}
when $c=A^Tv$, the solution is nonunique, the optimal value is $f_{min}=0$
\end{document}
