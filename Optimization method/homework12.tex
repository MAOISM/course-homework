%%%%%%%%%%%%%%%%%%%%%%%%%%%%%%%%%%%%%%%%%
% Short Sectioned Assignment
% LaTeX Template
% Version 1.0 (5/5/12)
%
% This template has been downloaded from:
% http://www.LaTeXTemplates.com
%
% Original author:
% Frits Wenneker (http://www.howtotex.com)
%
% License:
% CC BY-NC-SA 3.0 (http://creativecommons.org/licenses/by-nc-sa/3.0/)
%
%%%%%%%%%%%%%%%%%%%%%%%%%%%%%%%%%%%%%%%%%

%----------------------------------------------------------------------------------------
%	PACKAGES AND OTHER DOCUMENT CONFIGURATIONS
%----------------------------------------------------------------------------------------

\documentclass[paper=a4, fontsize=11pt]{scrartcl} % A4 paper and 11pt font size

\usepackage[T1]{fontenc} % Use 8-bit encoding that has 256 glyphs
\usepackage{fourier} % Use the Adobe Utopia font for the document - comment this line to return to the LaTeX default
\usepackage[english]{babel} % English language/hyphenation
\usepackage{amsmath,amsfonts,amsthm} % Math packages
\usepackage{pgfplots, tikz,comment}
\usetikzlibrary{intersections}
\usepackage[CJKbookmarks=true,
            colorlinks,linkcolor=black,anchorcolor=blue,citecolor=green]{hyperref}

\usepackage{lipsum} % Used for inserting dummy 'Lorem ipsum' text into the template

\usepackage{sectsty} % Allows customizing section commands
\allsectionsfont{\centering \normalfont\scshape} % Make all sections centered, the default font and small caps

\usepackage{fancyhdr} % Custom headers and footers
\pagestyle{fancyplain} % Makes all pages in the document conform to the custom headers and footers
\fancyhead{} % No page header - if you want one, create it in the same way as the footers below
\fancyfoot[L]{} % Empty left footer
\fancyfoot[C]{} % Empty center footer
\fancyfoot[R]{\thepage} % Page numbering for right footer
\renewcommand{\headrulewidth}{0pt} % Remove header underlines
\renewcommand{\footrulewidth}{0pt} % Remove footer underlines
\renewcommand\thesection{\roman{section}}

\setlength{\headheight}{13.6pt} % Customize the height of the header

\numberwithin{equation}{section} % Number equations within sections (i.e. 1.1, 1.2, 2.1, 2.2 instead of 1, 2, 3, 4)
\numberwithin{figure}{section} % Number figures within sections (i.e. 1.1, 1.2, 2.1, 2.2 instead of 1, 2, 3, 4)
\numberwithin{table}{section} % Number tables within sections (i.e. 1.1, 1.2, 2.1, 2.2 instead of 1, 2, 3, 4)

\setlength\parindent{0pt} % Removes all indentation from paragraphs - comment this line for an assignment with lots of text

%----------------------------------------------------------------------------------------
%	TITLE SECTION
%----------------------------------------------------------------------------------------

\newcommand{\horrule}[1]{\rule{\linewidth}{#1}} % Create horizontal rule command with 1 argument of height

\title{	
\normalfont \normalsize
\textsc{School of Software, Tsinghua University} \\ [25pt] % Your university, school and/or department name(s)
\horrule{0.5pt} \\[0.4cm] % Thin top horizontal rule
\huge Optimization Method\\ % The assignment title
\LARGE\textit{homework 12}
\horrule{2pt} \\[0.5cm] % Thick bottom horizontal rule
}

\author{Qingfu Wen \\ \normalsize 2015213495} % Your Info
\date{\normalsize\today} % Today's date or a custom date

\begin{document}

\maketitle % Print the title
\tableofcontents
\newpage
%----------------------------------------------------------------------------------------
%	PROBLEM 1
%----------------------------------------------------------------------------------------
\section{Problem 1}
Algorithm mapping is defined as follows:
\begin{equation} \nonumber
A(x)=
\left\{
\begin{aligned}
\left[\frac{3}{2}+\frac{1}{4}x,1+\frac{1}{2}x\right], when\quad x\geq2\\
\frac{1}{2}(x+1), when\quad x<2\\
\end{aligned}
\right.
\end{equation}
Prove that A is not closed at point $x=2$.
\\
\emph{\textbf{Proof:}}
we can choose point range $x^{(k)} = 2-\frac{1}{k}$, when $k\rightarrow\infty$, $x^{(k)}\rightarrow\hat{x}=2$.\\
From the definition of A(x) we have
\begin{equation} \nonumber
y^{(k)} = \frac{1}{2}(2-\frac{1}{k}+1) = \frac{3}{2}-\frac{1}{2k}
\end{equation}
When $k\rightarrow\infty$, $y^{(k)}\rightarrow\frac{3}{2}\notin A(\hat{x})=\{2\}$.\\
So $A(x)$ is not closed at point $x=2$. 
\section{Problem 2}
Define a algorithm mapping on set $X=[0,1]$.
\begin{equation} \nonumber
A(x)=
\left\{
\begin{aligned}
\left[0, x\right), 0<x\leq1 \\
0, x=0\\
\end{aligned}
\right.
\end{equation}
Discuss whether A is closed or not at: $x^{(1)}=0$, $x^{(2)}=\frac{1}{2}$.
\\
\emph{\textbf{Solution:}}\\
For $x^{(1)}=0$, we can choose point range $x^{(k)} = 0+\frac{1}{k}$, when $k\rightarrow\infty$, $x^{(k)}\rightarrow\hat{x}=0$.\\
From the definition of A(x) we have
\begin{equation} \nonumber
y^{(k)} = \left[0,\frac{1}{k}\right)
\end{equation}
When $k\rightarrow\infty$, $y^{(k)}\rightarrow0\in A(\hat{x})=A(0)=\{0\}$.\\
Since $A(x)$ is defined on $[0,1]$,we don't need to consider $0-\frac{1}{k}$, So $A(x)$ is closed at point $x^{(1)}=0$. \\
\\
For $x^{(2)}=0$, we can choose point range $x^{(k)} = \frac{1}{2}+\frac{1}{k}$, when $k\rightarrow\infty$, $x^{(k)}\rightarrow\hat{x}=\frac{1}{2}$.\\
From the definition of A(x) we have
\begin{equation} \nonumber
y^{(k)} = \left[0,\frac{1}{2}+\frac{1}{k}\right)
\end{equation}
When $k\rightarrow\infty$, $y^{(k)}\rightarrow[0,\frac{1}{2}] \notin A(\hat{x})=A(\frac{1}{2})=[0,\frac{1}{2})$.\\
So $A(x)$ is not closed at point $x^{(2)}=\frac{1}{2}$. 
\end{document}
