%%%%%%%%%%%%%%%%%%%%%%%%%%%%%%%%%%%%%%%%%
% Short Sectioned Assignment
% LaTeX Template
% Version 1.0 (5/5/12)
%
% This template has been downloaded from:
% http://www.LaTeXTemplates.com
%
% Original author:
% Frits Wenneker (http://www.howtotex.com)
%
% License:
% CC BY-NC-SA 3.0 (http://creativecommons.org/licenses/by-nc-sa/3.0/)
%
%%%%%%%%%%%%%%%%%%%%%%%%%%%%%%%%%%%%%%%%%

%----------------------------------------------------------------------------------------
%	PACKAGES AND OTHER DOCUMENT CONFIGURATIONS
%----------------------------------------------------------------------------------------

\documentclass[paper=a4, fontsize=11pt]{scrartcl} % A4 paper and 11pt font size

\usepackage[T1]{fontenc} % Use 8-bit encoding that has 256 glyphs
\usepackage{fourier} % Use the Adobe Utopia font for the document - comment this line to return to the LaTeX default
\usepackage[english]{babel} % English language/hyphenation
\usepackage{amsmath,amsfonts,amsthm} % Math packages
\usepackage{pgfplots, tikz,comment}
\usetikzlibrary{intersections}
\usepackage[CJKbookmarks=true,
            colorlinks,linkcolor=black,anchorcolor=blue,citecolor=green]{hyperref}

\usepackage{lipsum} % Used for inserting dummy 'Lorem ipsum' text into the template

\usepackage{sectsty} % Allows customizing section commands
\allsectionsfont{\centering \normalfont\scshape} % Make all sections centered, the default font and small caps

\usepackage{fancyhdr} % Custom headers and footers
\pagestyle{fancyplain} % Makes all pages in the document conform to the custom headers and footers
\fancyhead{} % No page header - if you want one, create it in the same way as the footers below
\fancyfoot[L]{} % Empty left footer
\fancyfoot[C]{} % Empty center footer
\fancyfoot[R]{\thepage} % Page numbering for right footer
\renewcommand{\headrulewidth}{0pt} % Remove header underlines
\renewcommand{\footrulewidth}{0pt} % Remove footer underlines
\renewcommand\thesection{\roman{section}}

\setlength{\headheight}{13.6pt} % Customize the height of the header

\numberwithin{equation}{section} % Number equations within sections (i.e. 1.1, 1.2, 2.1, 2.2 instead of 1, 2, 3, 4)
\numberwithin{figure}{section} % Number figures within sections (i.e. 1.1, 1.2, 2.1, 2.2 instead of 1, 2, 3, 4)
\numberwithin{table}{section} % Number tables within sections (i.e. 1.1, 1.2, 2.1, 2.2 instead of 1, 2, 3, 4)

\setlength\parindent{0pt} % Removes all indentation from paragraphs - comment this line for an assignment with lots of text

%----------------------------------------------------------------------------------------
%	TITLE SECTION
%----------------------------------------------------------------------------------------

\newcommand{\horrule}[1]{\rule{\linewidth}{#1}} % Create horizontal rule command with 1 argument of height

\title{	
\normalfont \normalsize
\textsc{School of Software, Tsinghua University} \\ [25pt] % Your university, school and/or department name(s)
\horrule{0.5pt} \\[0.4cm] % Thin top horizontal rule
\huge Optimization Method\\ % The assignment title
\LARGE\textit{homework 10}
\horrule{2pt} \\[0.5cm] % Thick bottom horizontal rule
}

\author{Qingfu Wen \\ \normalsize 2015213495} % Your Info
\date{\normalsize\today} % Today's date or a custom date

\begin{document}

\maketitle % Print the title
\tableofcontents
\newpage
%----------------------------------------------------------------------------------------
%	PROBLEM 1
%----------------------------------------------------------------------------------------
\section{Problem 1}
Give a function
\begin{equation}
f(x) = \frac{x_1+x_2}{3+x_1^2+x_2^2+x_1x_2}  \nonumber
\end{equation}
Please find minimum point.\\
\emph{\textbf{Solution:}}

\begin{equation}
\left\{
\begin{aligned}
\frac{\partial f}{\partial x_1} &= \frac{-x_1^2-2x_1x_2+3}{(3+x_1^2+x_2^2+x_1x_2)^2} =0\\
\frac{\partial f}{\partial x_2} &= \frac{-x_2^2-2x_1x_2+3}{(3+x_1^2+x_2^2+x_1x_2)^2} =0\\
\end{aligned} 
\right. \nonumber
\end{equation}
we can get arrest points $x^{(1)}=(1,1), x^{(2)}=(-1,-1)$.\\
\begin{align} \nonumber
\frac{\partial^2 f}{\partial x_1^2} &= \frac{-18x_1-12x_2+2x_1^3-2x_2^3+6x_1^2x_2}{(3+x_1^2+x_2^2+x_1x_2)^3}\\\nonumber
\frac{\partial^2 f}{\partial x_2^2} &= \frac{-12x_1-18x_2-2x_1^3+2x_2^3+6x_1x_2^2}{(3+x_1^2+x_2^2+x_1x_2)^3}\\\nonumber
\frac{\partial^2 f}{\partial x_1x_2} &= \frac{-12x_1-12x_2+6x_1^2x_2+6x_1x_2^2}{(3+x_1^2+x_2^2+x_1x_2)^3}\\\nonumber
\end{align}
$\nabla^2 f(x^{(1)}) = \begin{bmatrix} -\frac{1}{9} & -\frac{1}{18} \\ -\frac{1}{18} & -\frac{1}{9}\end{bmatrix}$ \qquad
$\nabla^2 f(x^{(2)}) = \begin{bmatrix} \frac{1}{9} & \frac{1}{18} \\ \frac{1}{18} & \frac{1}{9}\end{bmatrix}$ \\
Since $\nabla^2 f(x^{(1)})$ is a negative definite matrix and $\nabla^2 f(x^{(2)})$ is a positive definite matrix, $(-1, -1)$ is $f(x)'s$ minimum point.
\section{Problem 2}
Given a non-linear programming problem.
\begin{alignat}{2}          \nonumber
\min\quad & (x_1-\frac{9}{4})^2+(x_2-2)^2 \\    \nonumber
\mbox{s.t.}\quad            \nonumber
& -x_1^2+x_2 \geq 0\\        \nonumber
& x_1+x_2 \leq 6\\         \nonumber
& x_1,x_2 \geq 0
\end{alignat}
Judge the following point whether they are optimized solution or not.\\
\begin{equation}
x^{(1)}=\begin{bmatrix} \frac{3}{2} \\ \frac{9}{4}\end{bmatrix}, \quad
x^{(2)}=\begin{bmatrix} \frac{9}{4} \\ 2\end{bmatrix}, \quad
x^{(3)}=\begin{bmatrix} 0 \\ 2\end{bmatrix} \nonumber
\end{equation}
\emph{\textbf{Solution:}}\\
we can rewrite it into a convex programming:\\
\begin{alignat}{2}          \nonumber
\min\quad & (x_1-\frac{9}{4})^2+(x_2-2)^2 \\    \nonumber
\mbox{s.t.}\quad            \nonumber
& -x_1^2+x_2 \geq 0\\        \nonumber
& 6-x_1-x_2 \geq 0\\         \nonumber
& x_1,x_2 \geq 0
\end{alignat}
so we just need to check $x^{(1)}$, $x^{(2)}$, $x^{(3)}$ if they are KKT point.\\
for $x^{(1)}$, its KKT condition:\\
\begin{equation}
\left\{
\begin{aligned}
2(x_1-\frac{9}{4})+2w_1x_1 =0\\
2(x_2-2) - w_1 = 0\\
w_1 \geq 0\\
\end{aligned}
\right. \nonumber
\end{equation}
we can get $w_1=\frac{1}{2}$, $x^{(1)}$ is a optimized solution, optimized value is $\frac{5}{8}$.\\
for $x^{(2)}$, we can easily find $x^{(2)}$ is not a optimized solution.\\
for $x^{(3)}$, its KKT condition:\\ 
\begin{equation}
\left\{
\begin{aligned}
2(x_1-\frac{9}{4})-w_3 =0\\
2(x_2-2) = 0\\
w_3 \geq 0\\
\end{aligned}
\right. \nonumber
\end{equation}
This equation has no solution, so $x^{(3)}$ is not a optimized solution.
\section{Problem 3}
Please compute the minimum distance between base point $x^{(0)}=(0,0)^T$ and convex set 
\begin{equation}
S = {x|x_1+x_2\geq4,2x_1+x_2\geq5} \nonumber
\end{equation}
\emph{\textbf{Solution:}}\\
we can rewrite original problem into a convex programming:
\begin{alignat}{2}          \nonumber
\min\quad & x_1^2+x_2^2 \\    \nonumber
\mbox{s.t.}\quad            \nonumber
& x_1+x_2-4 \geq 0\\        \nonumber
& 2x_1+x_2-5 \geq 0\\         \nonumber
& x_1,x_2 \geq 0
\end{alignat}
KKT condition is as follow:
\begin{equation}
\left\{
\begin{aligned}
2x_1-w_1-2w_2 =0\\
2x_2-w_1-w_2 =0\\
w_1(x_1+x_2-4) =0\\
w_2(2x_1+x_2-5) =0\\
w_1, w_2\geq 0\\
x_1+x_2-4 \geq 0\\       
2x_1+x_2-5 \geq 0\\       
\end{aligned}
\right. \nonumber
\end{equation}
we can get the solution $\overline{x}=(2, 2)^T$, minimum distance is $2\sqrt{2}$.
\end{document}
