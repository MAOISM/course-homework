%%%%%%%%%%%%%%%%%%%%%%%%%%%%%%%%%%%%%%%%%
% Short Sectioned Assignment
% LaTeX Template
% Version 1.0 (5/5/12)
%
% This template has been downloaded from:
% http://www.LaTeXTemplates.com
%
% Original author:
% Frits Wenneker (http://www.howtotex.com)
%
% License:
% CC BY-NC-SA 3.0 (http://creativecommons.org/licenses/by-nc-sa/3.0/)
%
%%%%%%%%%%%%%%%%%%%%%%%%%%%%%%%%%%%%%%%%%

%----------------------------------------------------------------------------------------
%	PACKAGES AND OTHER DOCUMENT CONFIGURATIONS
%----------------------------------------------------------------------------------------

\documentclass[paper=a4, fontsize=11pt]{scrartcl} % A4 paper and 11pt font size

\usepackage[T1]{fontenc} % Use 8-bit encoding that has 256 glyphs
\usepackage{fourier} % Use the Adobe Utopia font for the document - comment this line to return to the LaTeX default
\usepackage[english]{babel} % English language/hyphenation
\usepackage{amsmath,amsfonts,amsthm} % Math packages
\usepackage{pgfplots, tikz}
\usetikzlibrary{intersections}
\usepackage[CJKbookmarks=true,
            colorlinks,linkcolor=black,anchorcolor=blue,citecolor=green]{hyperref}

\usepackage{lipsum} % Used for inserting dummy 'Lorem ipsum' text into the template

\usepackage{sectsty} % Allows customizing section commands
\allsectionsfont{\centering \normalfont\scshape} % Make all sections centered, the default font and small caps

\usepackage{fancyhdr} % Custom headers and footers
\pagestyle{fancyplain} % Makes all pages in the document conform to the custom headers and footers
\fancyhead{} % No page header - if you want one, create it in the same way as the footers below
\fancyfoot[L]{} % Empty left footer
\fancyfoot[C]{} % Empty center footer
\fancyfoot[R]{\thepage} % Page numbering for right footer
\renewcommand{\headrulewidth}{0pt} % Remove header underlines
\renewcommand{\footrulewidth}{0pt} % Remove footer underlines
\renewcommand\thesection{\roman{section}}

\setlength{\headheight}{13.6pt} % Customize the height of the header

\numberwithin{equation}{section} % Number equations within sections (i.e. 1.1, 1.2, 2.1, 2.2 instead of 1, 2, 3, 4)
\numberwithin{figure}{section} % Number figures within sections (i.e. 1.1, 1.2, 2.1, 2.2 instead of 1, 2, 3, 4)
\numberwithin{table}{section} % Number tables within sections (i.e. 1.1, 1.2, 2.1, 2.2 instead of 1, 2, 3, 4)

\setlength\parindent{0pt} % Removes all indentation from paragraphs - comment this line for an assignment with lots of text

%----------------------------------------------------------------------------------------
%	TITLE SECTION
%----------------------------------------------------------------------------------------

\newcommand{\horrule}[1]{\rule{\linewidth}{#1}} % Create horizontal rule command with 1 argument of height

\title{	
\normalfont \normalsize
\textsc{School of Software, Tsinghua University} \\ [25pt] % Your university, school and/or department name(s)
\horrule{0.5pt} \\[0.4cm] % Thin top horizontal rule
\huge Optimization Method\\ % The assignment title
\LARGE\textit{homework 3}
\horrule{2pt} \\[0.5cm] % Thick bottom horizontal rule
}

\author{Qingfu Wen \\ \normalsize 2015213495} % Your Info
\date{\normalsize\today} % Today's date or a custom date

\begin{document}

\maketitle % Print the title
\tableofcontents
\newpage
%----------------------------------------------------------------------------------------
%	PROBLEM 1
%----------------------------------------------------------------------------------------

\section{Problem 1}
solve the following LP problem using two-step algorithm.
\begin{enumerate}
\item
\begin{alignat}{2}          \nonumber
\max\quad & 3x_1-5x_2 \\    \nonumber
\mbox{s.t.}\quad            \nonumber
& -x_1+2x_2+4x_3 \leq 4\\        \nonumber
& x_1+x_2+2x_3 \leq 5\\         \nonumber
& -x_1+2x_2+x_3 \geq 1\\          \nonumber
& x_1,x_2,x_3 \geq 0
\end{alignat}
\emph{\textbf{Solution:}}\\

Lead slack variable $x_4$, $x_5$, $x_6$ into it.
\begin{alignat}{2}          \nonumber
\max\quad & 3x_1-5x_2 \\    \nonumber
\mbox{s.t.}\quad            \nonumber
& -x_1+2x_2+4x_3+x_4=4\\        \nonumber
& x_1+x_2+2x_3 +x_5=5\\         \nonumber
& -x_1+2x_2+x_3-x_6= 1\\          \nonumber
& x_j\geq 0, j=1,\cdots,6
\end{alignat}
solve it using two-step algorithm. Add artificial variable $x_7$.\\
\begin{alignat}{2}          \nonumber
\min\quad & x_7 \\    \nonumber
\mbox{s.t.}\quad            \nonumber
& -x_1+2x_2+4x_3+x_4=4\\        \nonumber
& x_1+x_2+2x_3 +x_5=5\\         \nonumber
& -x_1+2x_2+x_3-x_6+x_7= 1\\          \nonumber
& x_j\geq 0, j=1,\cdots,7
\end{alignat}

\begin{tabular}{|c|c|c|c|c|c|c|c|c|}
\hline &$x_1$&$x_2$&$x_3$&$x_4$&$x_5$&$x_6$&$x_7$&\\
\hline$x_4$&-1&2&4&1&0&0&0&4\\
$x_5$&1&1&2&0&1&0&0&5\\
$x_7$&-1&\Large{\textcircled{\small{2}}}&1&0&0&-1&1&1\\
\hline f&-1&2&1&0&0&-1&0&1\\
\hline
\hline $x_4$&0 & 0 & 3& 1& 0& 1&-1&3\\
$x_5$&$\frac{3}{2}$&0&$\frac{3}{2}$&0&1&$\frac{1}{2}$&-$\frac{1}{2}$&$\frac{9}{2}$\\
$x_2$&-$\frac{1}{2}$& 1&$\frac{1}{2}$& 0& 0&-$\frac{1}{2}$&$\frac{1}{2}$&$\frac{1}{2}$\\
\hline f    &0 & 0&0& 0& 0& 0& -1&0\\
\hline
\end{tabular}
\\\\
we get a basic feasible of original LP problem $\mathbf{x}=(0,\frac{1}{2},0,3,\frac{9}{2},0)$.

\begin{tabular}{|c|c|c|c|c|c|c|c|}
\hline &$x_1$&$x_2$&$x_3$&$x_4$&$x_5$&$x_6$&\\
\hline$x_4$&0&0&\Large{\textcircled{\small{3}}}&1&0&1&3\\
$x_5$&$\frac{3}{2}$&0&$\frac{3}{2}$&0&1&$\frac{1}{2}$&$\frac{9}{2}$\\
$x_2$&-$\frac{1}{2}$&1&$\frac{1}{2}$&0&0&-$\frac{1}{2}$&$\frac{1}{2}$\\
\hline f&-$\frac{1}{2}$&0&$-\frac{5}{2}$&0&0&$\frac{5}{2}$&$-\frac{5}{2}$\\
\hline
\hline $x_3$&0 & 0& 1& $\frac{1}{3}$& 0& $\frac{1}{3}$&1\\
$x_5$&\Large{\textcircled{\small{$\frac{3}{2}$}}}&0&0&-$\frac{1}{2}$&1&0&3\\
$x_2$&-$\frac{1}{2}$& 1& 0& -$\frac{1}{6}$&0&$-\frac{2}{3}$&0\\
\hline f    &$-\frac{1}{2}$ & 0&0&$\frac{5}{6}$& 0&$\frac{10}{3}$&0\\
\hline
\hline $x_3$&0 & 0& 1& $\frac{1}{3}$& 0& $\frac{1}{3}$&1\\
$x_1$&$1$&0&0&-$\frac{1}{3}$&$\frac{2}{3}$&0&2\\
$x_2$&0 & 1& 0& -$\frac{1}{3}$&$\frac{1}{3}$&$-\frac{2}{3}$&1\\
\hline f    & 0& 0&0&$\frac{2}{3}$&$\frac{1}{3}$&$\frac{10}{3}$&1\\
\hline
\end{tabular}
\\\\
From the above table, we can see that solution $\mathbf{x}$=(2,1,1,0,0),$f_{max}=1$.


\item
\begin{alignat}{2}          \nonumber
\min\quad & x_1-3x_2+x_3\\    \nonumber
\mbox{s.t.}\quad            \nonumber
& 2x_1-x_2+x_3 = 8\\        \nonumber
& 2x_1+x_2 \geq 2\\         \nonumber
& x_1+2x_2 \leq 10\\          \nonumber
& x_1,x_2,x_3\geq0
\end{alignat}

\emph{\textbf{Solution:}}\\
Lead slack variable $x_4$, $x_5$ into it.
\begin{alignat}{2}          \nonumber
\min\quad & x_1-3x_2+x_3\\    \nonumber
\mbox{s.t.}\quad            \nonumber
& 2x_1-x_2+x_3 = 8\\        \nonumber
& 2x_1+x_2-x_4 = 2\\         \nonumber
& x_1+2x_2+x_5 = 10\\          \nonumber
& x_j\geq0, j=1,\cdots,5
\end{alignat}
solve it using two-step algorithm. Add artificial variable $x_6$.\\
\begin{alignat}{2}          \nonumber
\min\quad & x_6\\    \nonumber
\mbox{s.t.}\quad            \nonumber
& 2x_1-x_2+x_3 = 8\\        \nonumber
& 2x_1+x_2-x_4+x_6 = 2\\         \nonumber
& x_1+2x_2+x_5 = 10\\          \nonumber
& x_j\geq0, j=1,\cdots,6
\end{alignat}

\begin{tabular}{|c|c|c|c|c|c|c|c|}
\hline &$x_1$&$x_2$&$x_3$&$x_4$&$x_5$&$x_6$&\\
\hline$x_3$&2&-1&1&0&0&0&8\\
$x_6$&\Large{\textcircled{\small{2}}}&1&0&-1&0&1&2\\
$x_5$&1&2&0&0&1&0&10\\
\hline f&2&1&0&0&0&0&2\\
\hline
\hline $x_3$&0 &-2& 1& 1& 0& -1&6\\
$x_1$&1&$\frac{1}{2}$&0&$-\frac{1}{2}$&0&$\frac{1}{2}$&1\\
$x_5$&0&\Large{\textcircled{\small{$\frac{3}{2}$}}}&0&$\frac{1}{2}$ & 1&-$\frac{1}{2}$&9\\
\hline f   & 0&0& 0& 0& 0& -1&0\\
\hline
\end{tabular}
\\\\
we get a basic feasible of original LP problem $\mathbf{x}=(1,0,6,0,9)$.

\begin{tabular}{|c|c|c|c|c|c|c|}
\hline &$x_1$&$x_2$&$x_3$&$x_4$&$x_5$&\\
\hline $x_3$&0 &-2& 1& 1& 0&6\\
$x_1$&1&$\frac{1}{2}$&0&$-\frac{1}{2}$&0&1\\
$x_5$&0&\Large{\textcircled{\small{$\frac{3}{2}$}}}&0&$\frac{1}{2}$ & 1&9\\
\hline f    &0 & $\frac{3}{2}$&0& $\frac{1}{2}$& 0&7\\
\hline
\hline &$x_1$&$x_2$&$x_3$&$x_4$&$x_5$&\\
\hline $x_3$&0 & 0& 1& $\frac{5}{3}$&$\frac{4}{3}$&18\\
$x_1$&1&0&0&$-\frac{2}{3}$&$-\frac{1}{3}$&-2\\
$x_2$&0&1&0&$\frac{1}{3}$&$\frac{2}{3}$&6\\
\hline f    &0 & 0&0& 0& -1&-2\\
\hline
\end{tabular}
\\\\
From the above table, we can see that solution $\mathbf{x}$=(-2,6,18,0,0),$f_{min}=-2$.

\end{enumerate}

\end{document}
