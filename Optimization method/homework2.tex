%%%%%%%%%%%%%%%%%%%%%%%%%%%%%%%%%%%%%%%%%
% Short Sectioned Assignment
% LaTeX Template
% Version 1.0 (5/5/12)
%
% This template has been downloaded from:
% http://www.LaTeXTemplates.com
%
% Original author:
% Frits Wenneker (http://www.howtotex.com)
%
% License:
% CC BY-NC-SA 3.0 (http://creativecommons.org/licenses/by-nc-sa/3.0/)
%
%%%%%%%%%%%%%%%%%%%%%%%%%%%%%%%%%%%%%%%%%

%----------------------------------------------------------------------------------------
%	PACKAGES AND OTHER DOCUMENT CONFIGURATIONS
%----------------------------------------------------------------------------------------

\documentclass[paper=a4, fontsize=11pt]{scrartcl} % A4 paper and 11pt font size

\usepackage[T1]{fontenc} % Use 8-bit encoding that has 256 glyphs
\usepackage{fourier} % Use the Adobe Utopia font for the document - comment this line to return to the LaTeX default
\usepackage[english]{babel} % English language/hyphenation
\usepackage{amsmath,amsfonts,amsthm} % Math packages
\usepackage{pgfplots, tikz}
\usetikzlibrary{intersections}
\usepackage[CJKbookmarks=true,
            colorlinks,linkcolor=black,anchorcolor=blue,citecolor=green]{hyperref}

\usepackage{lipsum} % Used for inserting dummy 'Lorem ipsum' text into the template

\usepackage{sectsty} % Allows customizing section commands
\allsectionsfont{\centering \normalfont\scshape} % Make all sections centered, the default font and small caps

\usepackage{fancyhdr} % Custom headers and footers
\pagestyle{fancyplain} % Makes all pages in the document conform to the custom headers and footers
\fancyhead{} % No page header - if you want one, create it in the same way as the footers below
\fancyfoot[L]{} % Empty left footer
\fancyfoot[C]{} % Empty center footer
\fancyfoot[R]{\thepage} % Page numbering for right footer
\renewcommand{\headrulewidth}{0pt} % Remove header underlines
\renewcommand{\footrulewidth}{0pt} % Remove footer underlines
\renewcommand\thesection{\roman{section}}

\setlength{\headheight}{13.6pt} % Customize the height of the header

\numberwithin{equation}{section} % Number equations within sections (i.e. 1.1, 1.2, 2.1, 2.2 instead of 1, 2, 3, 4)
\numberwithin{figure}{section} % Number figures within sections (i.e. 1.1, 1.2, 2.1, 2.2 instead of 1, 2, 3, 4)
\numberwithin{table}{section} % Number tables within sections (i.e. 1.1, 1.2, 2.1, 2.2 instead of 1, 2, 3, 4)

\setlength\parindent{0pt} % Removes all indentation from paragraphs - comment this line for an assignment with lots of text

%----------------------------------------------------------------------------------------
%	TITLE SECTION
%----------------------------------------------------------------------------------------

\newcommand{\horrule}[1]{\rule{\linewidth}{#1}} % Create horizontal rule command with 1 argument of height

\title{	
\normalfont \normalsize
\textsc{School of Software, Tsinghua University} \\ [25pt] % Your university, school and/or department name(s)
\horrule{0.5pt} \\[0.4cm] % Thin top horizontal rule
\huge Optimization Method\\ % The assignment title
\LARGE\textit{homework 2}
\horrule{2pt} \\[0.5cm] % Thick bottom horizontal rule
}

\author{Qingfu Wen \\ \normalsize 2015213495} % Your Info
\date{\normalsize\today} % Today's date or a custom date

\begin{document}

\maketitle % Print the title
\tableofcontents
\newpage
%----------------------------------------------------------------------------------------
%	PROBLEM 1
%----------------------------------------------------------------------------------------

\section{Problem 1}
Suppose $S = \{x|Ax\geq b\}$, $A$ is a $m*n$ matrix, $m > n$ and $r(A)=n$. Prove that the necessary and sufficient condition of $x^{(0)}$ is $S$'s pole is that $A$ and $b$ can be divided as follows.
\begin{equation}  \nonumber
A = \begin{bmatrix} A_1 \\  A_2 \end{bmatrix}\quad
b = \begin{bmatrix} b_1 \\ b_2 \end{bmatrix}
\end{equation}
Among them, $A_1$ has $n$ rows and $r(A)=n$, $b_1$ is a $n$-dimensional column vector, subject to $A_1x^{(0)}=b_1, A_2x^{(0)}\geq b_2$. \\
\emph{\textbf{Proof:}}\\
\begin{itemize}
\item prove the necessity first.\\
we know that $x^{(0)}$ is $S$'s pole. let me prove it by contradiction.Divide $A,b$ into the following form at $x^{(0)}$:
\begin{equation}  \nonumber
A = \begin{bmatrix} A_1 \\ A_2 \end{bmatrix}, \ b = \begin{bmatrix} b_1 \\ b_2 \end{bmatrix}, \ A_1x^{(0)} = b_1,\ A_2x^{(0)} > b_2.
\end{equation}
$r(A_1)= k < n$, suppose first $k$ columns is linearly independent, we can rewrite $A_1 = [B \ N]$, $B$ is invertible.
\begin{equation}  \nonumber
x = \begin{bmatrix} x_B \\ x_N \end{bmatrix}, \ x_B = B^{-1}b_1 - B^{-1}Nx_N.
\end{equation}
The solution of $A_1x=b_1$ is
\begin{equation}
x = \begin{bmatrix} x_B \\ x_N \end{bmatrix} = \begin{bmatrix} B^{-1}b_1-B^{-1}Nx_N \\ x_N \end{bmatrix}.
\end{equation}
$S$'s pole
\begin{equation}
x^{(0)} = \begin{bmatrix} x^{(0)}_B \\ x^{(0)}_N \end{bmatrix} = \begin{bmatrix} B^{-1}b_1-B^{-1}Nx^{(0)}_N \\ x^{(0)}_N \end{bmatrix}.
\end{equation}
Since $ A_2x^{(0)} > b_2$, $\exists x_N \in N_{\delta}(x^{(0)}_N)$, $A_1x=b_1$ and $A_2x>b_2$ holds. On the line passed $x^{(0)}_N$, $\exists x^{(1)}_N, x^{(2)}_N \in N_{\delta}(x^{(0)}_N)$, subject to $\lambda x^{(1)}_N+(1-\lambda)x^{(2)}_N = x^{(0)}_N, \lambda \in (0,1)$, substitute this into (i.2), we get
\begin{equation}
\begin{array}{rcl}
x^{(0)}&=&\begin{bmatrix} B^{-1}b_1-B^{-1}N(\lambda x^{(1)}_N+(1-\lambda)x^{(2)}_N) \\ \lambda x^{(1)}_N+(1-\lambda)x^{(2)}_N \end{bmatrix}\\
&=&\lambda\begin{bmatrix} B^{-1}b_1-B^{-1}Nx^{(1)}_N \\ x^{(1)}_N \end{bmatrix}+(1-\lambda)\begin{bmatrix} B^{-1}b_1-B^{-1}Nx^{(2)}_N \\ x^{(2)}_N \end{bmatrix}.
\end{array}
\end{equation}
Thus, $x^{(0)}$ can be represented as convex combination of two point in $S$ which is contradicted with $x^{(0)}$ is pole.

\item Then prove the sufficiency.\\
Suppose at point $x^{(0)}$, $A$, $b$ can be divided as follows:
\begin{equation}  \nonumber
A = \begin{bmatrix} A_1 \\ A_2 \end{bmatrix}, \ b = \begin{bmatrix} b_1 \\ b_2 \end{bmatrix}, \ A_1x^{(0)} = b_1,\ A_2x^{(0)} \geq b_2, r(A_1)=n.
\end{equation}
And suppose $\exists x^{(1)}, x^{(2)} \in S$, subject to
\begin{equation}  \nonumber
\begin{array}{rcl}
x^{(0)} &=& \lambda x^{(1)} +(1-\lambda)x^{(2)}, \lambda \in (0,1)\\
A_1x^{(0)}&=& \lambda A_1x^{(1)} +(1-\lambda)A_1x^{(2)}
\end{array}
\end{equation}
Because $A_1x^{(0)}=b_1$, $A_1x^{(1)}\geq b_1$, $A_1x^{(2)}\geq b_1$and $\lambda,1-\lambda>0$, substitute it into the above equation.
\begin{equation}  \nonumber
b_1 = A_1x^{(0)} = \lambda A_1x^{(1)}+(1-\lambda)A_1x^{(2)} \geq \lambda b_1+(1-\lambda)b_1=b_1
\end{equation}
so we have
\begin{equation}  \nonumber
\begin{array}{rcl}
\lambda A_1x^{(1)}+(1-\lambda)A_1x^{(2)} &=&\lambda b_1+(1-\lambda)b_1 \\
\lambda(A_1x^{(1)}-b_1)+(1-\lambda)(A_1x^{(2)}-b_1)&=&0
\end{array}
\end{equation}
Since $\lambda, 1-\lambda>0$, $A_1x^{(1)}-b_1\geq0$, $A_1x^{(2)}-b_1\geq0$, so $A_1x^{(1)}-b_1=0$, $A_1x^{(2)}-b_1=0$, then 
\begin{equation}  \nonumber
\begin{array}{rcl}
A_1x^{(0)}=A_1x^{(1)}=A_1x^{(2)}=b_1\\
x^{(0)}=x^{(1)}=x^{(2)}
\end{array}
\end{equation}
Thus, $x^{(0)}$ is pole.
\end{itemize}
\section{Problem 2}
solve the following LP problem using simplex algorithm.
\begin{enumerate}
\item
\begin{alignat}{2}          \nonumber
\min\quad & 3x_1-5x_2-2x_3-x_4\\    \nonumber
\mbox{s.t.}\quad            \nonumber
& x_1+x_2+x_3 \leq 4\\        \nonumber
& 4x_1-x_2+x_3+2x_4 \leq 6\\         \nonumber
& -x_1+x_2+2x_3+3x_4 \leq 12\\          \nonumber
& x_j \geq 0, j=1,\cdots,4
\end{alignat}
\emph{\textbf{Solution:}}\\
Lead slack variable $x_5$, $x_6$, $x_7$.
\begin{alignat}{2}          \nonumber
\min\quad & 3x_1-5x_2-2x_3-x_4\\    \nonumber
\mbox{s.t.}\quad            \nonumber
& x_1+x_2+x_3+x_5=4\\        \nonumber
& 4x_1-x_2+x_3+2x_4+x_6 = 6\\         \nonumber
& -x_1+x_2+2x_3+3x_4+x_7 = 12\\          \nonumber
& x_j \geq 0, j=1,\cdots,7
\end{alignat}
solve it using simplex algorithm.\\
\begin{tabular}{|c|c|c|c|c|c|c|c|c|}
\hline &$x_1$&$x_2$&$x_3$&$x_4$&$x_5$&$x_6$&$x_7$&\\
\hline$x_5$&1&1&1&0&1&0&0&4\\
$x_6$&4&-1&1&2&0&1&0&6\\
$x_7$&-1&1&2&3&0&0&1&12\\
\hline f&-3&5&2&1&0&0&0&0\\
\hline
\hline $x_2$&1 & 1& 1& 0& 1& 0& 0&4\\
$x_6$&5 & 0& 2& 2& 1& 1& 0&10\\
$x_7$&-2& 0& 1& 3& -1& 0& 1&8\\
\hline f    &8 & 0&-3& 1& -5& 0& 0&-20\\
\hline
\hline $x_2$&1 & 1& 1& 0& 1& 0& 0&4\\
$x_6$&$\frac{19}{3}$ & 0& $-\frac{4}{3}$& 0& 2& 1&$-\frac{2}{3}$&$\frac{14}{3}$\\
$x_4$&$-\frac{2}{3}$ & 0& $\frac{1}{3}$& 1& 0& 0& $\frac{1}{3}$&$\frac{8}{3}$\\
\hline f    &$-\frac{22}{3}$ & 0&$-\frac{10}{3}$& 0&$-\frac{14}{3}$& 0& $-\frac{1}{3}$&$-\frac{68}{3}$\\
\hline
\end{tabular}
\\\\
From the above simplex table, we can see that solution $\mathbf{x}$=(0,4,0,$\frac{8}{3}$,0,$\frac{14}{3}$,0),$f_{min}=-\frac{68}{3}$.


\item
\begin{alignat}{2}          \nonumber
\min\quad & -3x_1-x_2\\    \nonumber
\mbox{s.t.}\quad            \nonumber
& 3x_1+3x_2+x_3 = 30\\        \nonumber
& 4x_1-4x_2+x_4 = 6\\         \nonumber
& 2x_1-x_2 \leq 12\\          \nonumber
& x_j \geq 0, j=1,\cdots,4
\end{alignat}

\emph{\textbf{Solution:}}\\
Lead slack variable $x_5$.
\begin{alignat}{2}          \nonumber
\min\quad & -3x_1-x_2\\    \nonumber
\mbox{s.t.}\quad            \nonumber
& 3x_1+3x_2+x_3 = 30\\        \nonumber
& 4x_1-4x_2+x_4 = 6\\         \nonumber
& 2x_1-x_2+x_5 = 12\\          \nonumber
& x_j \geq 0, j=1,\cdots,5
\end{alignat}
solve it using simplex algorithm.\\
\begin{tabular}{|c|c|c|c|c|c|c|}
\hline &$x_1$&$x_2$&$x_3$&$x_4$&$x_5$&\\
\hline$x_3$&3&3&1&0&0&30\\
$x_4$&4&-4&0&1&0&16\\
$x_5$&2&-1&0&0&1&12\\
\hline f&3&1&0&0&0&0\\
\hline
\hline$x_3$&0&6&1&0&0&18\\
$x_1$&1&-1&0&$-\frac{3}{4}$&0&4\\
$x_5$&0&1&0&$\frac{1}{4}$&1&4\\
\hline f&0&4&0&$-\frac{1}{2}$&0&-12\\
\hline
\hline$x_2$&0&1&$\frac{1}{6}$&$-\frac{1}{8}$&0&3\\
$x_1$&1&0&$\frac{1}{6}$&$\frac{1}{8}$&0&7\\
$x_5$&0&0&$-\frac{1}{6}$&$-\frac{3}{8}$&1&1\\
\hline f&0&0&$-\frac{2}{3}$&$-\frac{1}{4}$&0&-24\\
\hline
\end{tabular}
\\\\
From the above simplex table, we can see that solution $\mathbf{x}$=(7,3,0,0,1),$f_{min}=-24$.
\end{enumerate}

\section{Problem 3}
Suppose when solving LP problem using simplex algorithm
\begin{alignat}{2}  \nonumber
\min\quad & \mathbf{cx}\\    \nonumber
\mbox{s.t.}\quad   \nonumber
& \mathbf{Ax=b}\\           \nonumber
& \mathbf{x}\geq \mathbf{0}\\         \nonumber
\end{alignat}
in one iteration, the check number of variable $x_j$ $z_j-c_j>0$, and the corresponding column in simplex table $y_j=\mathbf{B}^{-1}\mathbf{p}_j\leq0$, Prove that
\begin{equation}  \nonumber
\mathbf{d} = \begin{bmatrix}
    -y_j \\  0 \\ \vdots \\ 1 \\ \vdots \\ 0
    \end{bmatrix}
\end{equation}
is  the polar direction of feasible region, component 1 correspond to $x_j$ among it. (suppose $\mathbf{B}$ is $\mathbf{A}$'s first $m$ columns)\\
\emph{\textbf{Solution:}}\\
Suppose $\mathbf{A}$ is a $m*n$ matrix, and
\begin{equation}  \nonumber
\mathbf{A} = \begin{bmatrix}\mathbf{p}_1 \ \mathbf{p}_2 \ \cdots \ \mathbf{p}_m \ \cdots \ \mathbf{p}_n \end{bmatrix}
= \begin{bmatrix}\mathbf{B}\ \mathbf{p}_{m+1} \ \cdots \ \mathbf{p}_n \ \end{bmatrix}
\end{equation}
Since
\begin{equation}  \nonumber
\mathbf{Ad} = \begin{bmatrix}\mathbf{B}\ \mathbf{p}_{m+1} \ \cdots \ \mathbf{p}_n \ \end{bmatrix}
\begin{bmatrix}-\mathbf{B}^{-1}\mathbf{p}_j \\ 0 \\ \vdots \\ 1 \\ \vdots \\ 0 \end{bmatrix} = -\mathbf{p}_j+\mathbf{p}_j = 0
\end{equation}
and $\mathbf{d}\geq0$, $\mathbf{d}$ is the direction of feasible region. Then, we can prove $\mathbf{d}$ is a polar direction by contradiction.\\
Suppose $\mathbf{d}$ is a positive linear combination of two directions of feasible region, $\mathbf{d_1}$, $\mathbf{d_2}$.
\begin{equation}
\mathbf{d} = \lambda\mathbf{d_1}+\mu\mathbf{d_2}
\end{equation}
$\lambda,\mu>0$, $\mathbf{d_1},\mathbf{d_2}\geq0$, from $\mathbf{d}$'s form, we can know that
\begin{equation}  \nonumber
\mathbf{d_1} = \begin{bmatrix}-y^{(1)}_{j} \\  0 \\ \vdots \\ a_j \\ \vdots \\ 0 \end{bmatrix},
\mathbf{d_2} = \begin{bmatrix}-y^{(2)}_{j} \\  0 \\ \vdots \\ b_j \\ \vdots \\ 0 \end{bmatrix}, a_j, b_j>0.
\end{equation}
Because $\mathbf{d_1}$ is the direction of feasible region, $\mathbf{Ad_1}=0$,$\mathbf{d_1}\geq0$.
\begin{equation}
-\mathbf{B}y^{(1)}_{j}+a_jp_j = \mathbf{0}
\end{equation}
In the same way,
\begin{equation}
-\mathbf{B}y^{(2)}_{j}+b_jp_j = \mathbf{0}
\end{equation}
from the above two equation, we can get
\begin{equation}
\begin{array}{rcl}
\frac{1}{a_j}\mathbf{B}y^{(1)}_{j} &=& \frac{1}{b_j}\mathbf{B}y^{(2)}_{j} \\
y^{(1)}_{j} &=& \frac{a_j}{b_j}y^{(2)}_{j}
\end{array}
\end{equation}
substitute it into $\mathbf{d_1}$, now we get
\begin{equation}
\mathbf{d_2} = \frac{b_j}{a_j}\mathbf{d_1}, a_j,b_j>0.
\end{equation}
$\mathbf{d_1}, \mathbf{d_2}$ are non-zero vectors with same direction, which is contradict with the hypothesis. Thus, $\mathbf{d}$ can not be represented as a positive linear combination of two directions of feasible region. $\mathbf{d}$ is polar direction.
\end{document}
