%%%%%%%%%%%%%%%%%%%%%%%%%%%%%%%%%%%%%%%%%
% Short Sectioned Assignment
% LaTeX Template
% Version 1.0 (5/5/12)
%
% This template has been downloaded from:
% http://www.LaTeXTemplates.com
%
% Original author:
% Frits Wenneker (http://www.howtotex.com)
%
% License:
% CC BY-NC-SA 3.0 (http://creativecommons.org/licenses/by-nc-sa/3.0/)
%
%%%%%%%%%%%%%%%%%%%%%%%%%%%%%%%%%%%%%%%%%

%----------------------------------------------------------------------------------------
%	PACKAGES AND OTHER DOCUMENT CONFIGURATIONS
%----------------------------------------------------------------------------------------

\documentclass[paper=a4, fontsize=11pt]{scrartcl} % A4 paper and 11pt font size

\usepackage[T1]{fontenc} % Use 8-bit encoding that has 256 glyphs
\usepackage{fourier} % Use the Adobe Utopia font for the document - comment this line to return to the LaTeX default
\usepackage[english]{babel} % English language/hyphenation
\usepackage{amsmath,amsfonts,amsthm} % Math packages
\usepackage{pgfplots, tikz,comment}
\usetikzlibrary{intersections}
\usepackage[CJKbookmarks=true,
            colorlinks,linkcolor=black,anchorcolor=blue,citecolor=green]{hyperref}

\usepackage{lipsum} % Used for inserting dummy 'Lorem ipsum' text into the template

\usepackage{sectsty} % Allows customizing section commands
\allsectionsfont{\centering \normalfont\scshape} % Make all sections centered, the default font and small caps

\usepackage{fancyhdr} % Custom headers and footers
\pagestyle{fancyplain} % Makes all pages in the document conform to the custom headers and footers
\fancyhead{} % No page header - if you want one, create it in the same way as the footers below
\fancyfoot[L]{} % Empty left footer
\fancyfoot[C]{} % Empty center footer
\fancyfoot[R]{\thepage} % Page numbering for right footer
\renewcommand{\headrulewidth}{0pt} % Remove header underlines
\renewcommand{\footrulewidth}{0pt} % Remove footer underlines
\renewcommand\thesection{\roman{section}}

\setlength{\headheight}{13.6pt} % Customize the height of the header

\numberwithin{equation}{section} % Number equations within sections (i.e. 1.1, 1.2, 2.1, 2.2 instead of 1, 2, 3, 4)
\numberwithin{figure}{section} % Number figures within sections (i.e. 1.1, 1.2, 2.1, 2.2 instead of 1, 2, 3, 4)
\numberwithin{table}{section} % Number tables within sections (i.e. 1.1, 1.2, 2.1, 2.2 instead of 1, 2, 3, 4)

\setlength\parindent{0pt} % Removes all indentation from paragraphs - comment this line for an assignment with lots of text

%----------------------------------------------------------------------------------------
%	TITLE SECTION
%----------------------------------------------------------------------------------------

\newcommand{\horrule}[1]{\rule{\linewidth}{#1}} % Create horizontal rule command with 1 argument of height

\title{	
\normalfont \normalsize
\textsc{School of Software, Tsinghua University} \\ [25pt] % Your university, school and/or department name(s)
\horrule{0.5pt} \\[0.4cm] % Thin top horizontal rule
\huge Optimization Method\\ % The assignment title
\LARGE\textit{homework 9}
\horrule{2pt} \\[0.5cm] % Thick bottom horizontal rule
}

\author{Qingfu Wen \\ \normalsize 2015213495} % Your Info
\date{\normalsize\today} % Today's date or a custom date

\begin{document}

\maketitle % Print the title
\tableofcontents
\newpage
%----------------------------------------------------------------------------------------
%	PROBLEM 1
%----------------------------------------------------------------------------------------
\section{Problem 1}
Suppose $A$ is a $m*n$ matrix, $B$ is a $n*l$ matrix, $c\in E^n$, Prove that only one of the following two system has solutions:
\begin{enumerate}
\item $Ax\leq0,Bx=0,c^Tx>0, x\in E^n$
\item $A^Ty+B^Tz=c,y\geq0, y,z\in E^n$
\end{enumerate}
\emph{\textbf{Proof:}}
since $Bx=0$ is the same as
\begin{equation}
\left\{
\begin{aligned}
Bx\leq0 \\
Bx\geq0 \\
\end{aligned}
\right.\nonumber
\end{equation}
so the first system can turn into
\begin{equation} 
\begin{bmatrix}A \\ B \\ -B\end{bmatrix}x\leq0,c^Tx>0   \nonumber
\end{equation}
according to Farkas Theorem,
\begin{equation}
\begin{bmatrix}A^T & B^T & -B^T\end{bmatrix}\begin{bmatrix}y \\ u \\ v\end{bmatrix}=c, \begin{bmatrix}y \\ u \\ v\end{bmatrix}>0    \nonumber
\end{equation}
has no solutions. we set $z=u-v$, we know that $A^Ty+B^Tz=c,y\geq0$ has no solution, or vice versa.
\section{Problem 2}
Suppose $A$ is a $m*n$ matrix, $c\in E^n$, Prove that only one of the following two system has solutions:
\begin{enumerate}
\item $Ax\leq0,x\geq0,c^Tx>0, x\in E^n$
\item $A^Ty\geq0,y\geq0, y\in E^n$
\end{enumerate}

\emph{\textbf{Proof:}}\\
the first system can turn into
\begin{equation}
\begin{bmatrix}A \\ -I\end{bmatrix}x\leq0,c^Tx>0   \nonumber
\end{equation}
according to Farkas Theorem,
\begin{equation}
\begin{bmatrix}A^T & -I \end{bmatrix}\begin{bmatrix}y \\ u\end{bmatrix}=c, \begin{bmatrix}y \\ u\end{bmatrix}>0    \nonumber
\end{equation}
has no solutions. It means $A^Ty-u=c,y\geq0,u\geq0$ has no solution. Also we have
\begin{equation}
A^Ty\geq0,y\geq0 \nonumber
\end{equation}
has no solutions, or vice versa. 

\section{Problem 3}
$f(x_1,x_2)=10-2(x_2-x_1^2)^2$, $S=\{(x_1,x_2)|-11\leq x_1\leq1, -1\leq x_2\leq1\}$,is $f(x_1,x_2)$ a convex function on $S$?\\
\emph{\textbf{Solution:}}\\
for each ($x_1$,$x_2$),
\begin{equation}
\nabla^2f(x) = \begin{bmatrix} -24x_1^2+8x_2 & 8x_1 \\ 8x_1 &  -4 \end{bmatrix} \nonumber
\end{equation}
obviously, $\nabla^2f(x)$ is not positive semidefinite for all $x$, $f(x_1,x_2)$ is not a convex function.

\section{Problem 4}
Suppose $f$ is a convex function on $E^n$, $x^{(1)},x^{(2)},\cdots,x^{(n)}$belongs to $E^n$,prove that
\begin{equation}
f(\lambda_1x^{(1)}+\cdots+\lambda_kx^{(k)})\leq\lambda_1f(x^{(1)})+\cdots+\lambda_kf(x^{(k)}) \nonumber
\end{equation}
$\lambda_1+\lambda_2+\cdots+\lambda_k=1, \lambda_i>0, i= 1,2,\cdots,k$.\\
\emph{\textbf{Proof:}}\\
Using mathematical induction,when $k=2$, since $f(x)$ is a convex function on $E^n$.
\begin{equation}
f(\lambda_1x^{(1)}+\lambda_2x^{(2)})\leq \lambda_1f(x^{(1)})+ \lambda_2f(x^{(2)}) \nonumber
\end{equation}
Suppose when $k = n$ the equation holds, $k=n+1$,
\begin{equation}
f(\lambda_1x^{(1)}+\cdots+\lambda_{n+1}x^{(n+1)})= f(\sum_1^n\lambda_i(\frac{\lambda_1}{\sum_1^n\lambda_i}x^{(1)}+\cdots+\frac{\lambda_n}{\sum_1^n\lambda_i}x^{(n)})+\lambda_{n+1}x^{(n+1)}) \nonumber
\end{equation}
Set $\hat{x} = \frac{\lambda_1}{\sum_1^n\lambda_i}x^{(1)}+\cdots+\frac{\lambda_n}{\sum_1^n\lambda_i}x^{(n)}$, since $f (x)$ is convex and $\sum_1^n\lambda_i+\lambda_{n+1}=1$, we have
\begin{equation}
f(\lambda_1x^{(1)}+\cdots+\lambda_{n+1}x^{(n+1)})\leq (\sum_1^n\lambda_i)f(\hat{x})+\lambda_{n+1}f(x^{(n+1)})\nonumber
\end{equation}
Moreover, from hypothesis we have
\begin{equation}
f(\hat{x})\leq \frac{\lambda_1}{\sum_1^n\lambda_i}f(x^{(1)})+\cdots+ \frac{\lambda_n}{\sum_1^n\lambda_i}f(x^{(n)})\nonumber
\end{equation}
then we get
\begin{equation}
f(\lambda_1x^{(1)}+\lambda_2x^{(2)}+\cdots+\lambda_{n+1}x^{(n+1)})\leq \lambda_1f(x^{(1)})+\lambda_2f(x^{(2)})+\cdots+\lambda_{n+1}f(x^{(n+1)})\nonumber
\end{equation}
thus, the original equation holds.
\end{document}
