%%%%%%%%%%%%%%%%%%%%%%%%%%%%%%%%%%%%%%%%%
% Short Sectioned Assignment
% LaTeX Template
% Version 1.0 (5/5/12)
%
% This template has been downloaded from:
% http://www.LaTeXTemplates.com
%
% Original author:
% Frits Wenneker (http://www.howtotex.com)
%
% License:
% CC BY-NC-SA 3.0 (http://creativecommons.org/licenses/by-nc-sa/3.0/)
%
%%%%%%%%%%%%%%%%%%%%%%%%%%%%%%%%%%%%%%%%%

%----------------------------------------------------------------------------------------
%	PACKAGES AND OTHER DOCUMENT CONFIGURATIONS
%----------------------------------------------------------------------------------------

\documentclass[paper=a4, fontsize=11pt]{scrartcl} % A4 paper and 11pt font size

\usepackage[T1]{fontenc} % Use 8-bit encoding that has 256 glyphs
\usepackage{fourier} % Use the Adobe Utopia font for the document - comment this line to return to the LaTeX default
\usepackage[english]{babel} % English language/hyphenation
\usepackage{amsmath,amsfonts,amsthm} % Math packages
\usepackage{pgfplots, tikz,comment}
\usetikzlibrary{intersections}
\usepackage[CJKbookmarks=true,
            colorlinks,linkcolor=black,anchorcolor=blue,citecolor=green]{hyperref}

\usepackage{lipsum} % Used for inserting dummy 'Lorem ipsum' text into the template

\usepackage{sectsty} % Allows customizing section commands
\allsectionsfont{\centering \normalfont\scshape} % Make all sections centered, the default font and small caps

\usepackage{fancyhdr} % Custom headers and footers
\pagestyle{fancyplain} % Makes all pages in the document conform to the custom headers and footers
\fancyhead{} % No page header - if you want one, create it in the same way as the footers below
\fancyfoot[L]{} % Empty left footer
\fancyfoot[C]{} % Empty center footer
\fancyfoot[R]{\thepage} % Page numbering for right footer
\renewcommand{\headrulewidth}{0pt} % Remove header underlines
\renewcommand{\footrulewidth}{0pt} % Remove footer underlines
\renewcommand\thesection{\roman{section}}

\setlength{\headheight}{13.6pt} % Customize the height of the header

\numberwithin{equation}{section} % Number equations within sections (i.e. 1.1, 1.2, 2.1, 2.2 instead of 1, 2, 3, 4)
\numberwithin{figure}{section} % Number figures within sections (i.e. 1.1, 1.2, 2.1, 2.2 instead of 1, 2, 3, 4)
\numberwithin{table}{section} % Number tables within sections (i.e. 1.1, 1.2, 2.1, 2.2 instead of 1, 2, 3, 4)

\setlength\parindent{0pt} % Removes all indentation from paragraphs - comment this line for an assignment with lots of text

%----------------------------------------------------------------------------------------
%	TITLE SECTION
%----------------------------------------------------------------------------------------

\newcommand{\horrule}[1]{\rule{\linewidth}{#1}} % Create horizontal rule command with 1 argument of height

\title{	
\normalfont \normalsize
\textsc{School of Software, Tsinghua University} \\ [25pt] % Your university, school and/or department name(s)
\horrule{0.5pt} \\[0.4cm] % Thin top horizontal rule
\huge Optimization Method\\ % The assignment title
\LARGE\textit{homework 15}
\horrule{2pt} \\[0.5cm] % Thick bottom horizontal rule
}

\author{Qingfu Wen \\ \normalsize 2015213495} % Your Info
\date{\normalsize\today} % Today's date or a custom date

\begin{document}

\maketitle % Print the title
\tableofcontents
\newpage
%----------------------------------------------------------------------------------------
%	PROBLEM 1
%----------------------------------------------------------------------------------------
\section{Problem 1}
Consider the following problem:
\begin{alignat}{2}          \nonumber
\min\quad & x_1^3+x_2^3 \\    \nonumber
\mbox{s.t.}\quad            \nonumber
& x_1+x_2=1 \\    \nonumber
\end{alignat}\\
\begin{enumerate}
\item Solve the optimum of problem.
\item Define penalty function 
\begin{equation}\nonumber
F(x,\sigma) = x_1^3+x_2^3+\sigma(x_1+x_2-1)^2
\end{equation}
Can we get the optimal solution of original problem by solving non-constraint problem $\min F(x,\sigma)$? Why?
\end{enumerate}
~\\
\emph{\textbf{Solution:}}\\
\begin{enumerate}
\item we can use $x_2 = 1-x_1$ replace into objective function, we change original problem into a non-constraint problem
\begin{alignat}{2}          \nonumber
\min\quad & 3x_1^2-3x_1+1 \\    \nonumber
\end{alignat}
Set $f'(x)=6x-3=0$, we get $x=\frac{1}{2}$. Then we can easily get the optimal solution $\bar{x}=(\frac{1}{2},\frac{1}{2})^T$,$f_{min}=\frac{1}{4}$.
\item No, $F(x^{(k)},\sigma_k)\leq f(\bar{x})$ 's solution is not contained in a compact set.
\end{enumerate}
\section{Problem 2}
Consider the following problem:
\begin{alignat}{2}          \nonumber
\min\quad & x_1x_2 \\    \nonumber
\mbox{s.t.}\quad            \nonumber
& g(x)=-2x_1+x_2+3\geq0\\    \nonumber
\end{alignat}
\begin{enumerate}
\item  Prove that 
\begin{equation} \nonumber
\bar{x} = \begin{bmatrix}\begin{smallmatrix}\frac{3}{4}\\-\frac{3}{2}\end{smallmatrix}\end{bmatrix}
\end{equation}
is local optimum, and tell whether it is global optimum or not.
\item Define barrier function 
\begin{equation}\nonumber
G(x,r)=x_1x_2-r\ln g(x)
\end{equation}
\end{enumerate}
Solve this problem using inner point method and explain that sequence generated by inner point method incline to $\bar{x}$.\\~\\
\emph{\textbf{Solution:}}\\~\\
\begin{enumerate}
\item  at point $\bar{x}$,we have
\begin{equation}\nonumber
\nabla f(x) = \begin{bmatrix}x_2\\x_1\end{bmatrix} = \begin{bmatrix}-\frac{3}{2} \\ \frac{3}{4}\end{bmatrix}, \nabla g(\bar{x}) = \begin{bmatrix}-2 \\ 1\end{bmatrix}
\end{equation}
\begin{equation}\nonumber
\begin{bmatrix}-\frac{3}{2} \\ \frac{3}{4}\end{bmatrix}- w\begin{bmatrix}-2 \\ 1\end{bmatrix}=\begin{bmatrix}0\\ 0\end{bmatrix}
\end{equation}
Solve it and we get $w=\frac{3}{4}>0$, $\bar{x}$ is a KTT point.
we get Lagrange function
\begin{equation}\nonumber
L(x,w) = x_1x_2-w(-2x_1+x_2+3)
\end{equation}
Then
\begin{equation}\nonumber
\nabla^2L(x,w) = \begin{bmatrix}0 &1 \\ 1 & 0\end{bmatrix}
\end{equation}
Set 
\begin{equation}\nonumber
\nabla g(\bar{x})^Td=[-2, 1]\begin{bmatrix}d_1\\d_2\end{bmatrix}=0
\end{equation}
we get $d_2=2d_1$. $\forall d\in G$, we have
\begin{equation}\nonumber
d^T\nabla^2L(x,w)d=4d_1^2 > 0
\end{equation}
So, $\bar{x}$ is strict local optimum, and it is easy to find out $\bar{x}$ is not global optimum.
\item
\begin{equation}\nonumber
G(x,r)=x_1x_2-r\ln (-2x_1+x_2+3)
\end{equation}
Set 
\begin{equation} \nonumber
\left\{
\begin{aligned}
\frac{\partial G(x,r)}{\partial x_1}=x_2+\frac{2r}{-2x_1+x_2+3}=0\\
\frac{\partial G(x,r)}{\partial x_2}=x_1-\frac{r}{-2x_1+x_2+3}=0
\end{aligned}
\right.
\end{equation}
Solve the above equation, we get
\begin{equation} \nonumber
\left\{
\begin{aligned}
x_1 = \frac{3+\sqrt{9-16r}}{8}=0\\
x_2 = -\frac{3+\sqrt{9-16r}}{4}=0
\end{aligned}
\right.
\end{equation}
Set $r\rightarrow0$, we have 
\begin{equation}\nonumber
\bar{x} = (\frac{3}{4}, -\frac{3}{2})^T
\end{equation}
\end{enumerate}

\end{document}
